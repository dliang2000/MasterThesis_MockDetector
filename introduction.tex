%======================================================================
\chapter[Introduction]{Introduction\footnote{The contents of this 
		chapter have been incorporated within a paper that has been submitted for publication. Qian Liang and Patrick Lam, “MockDetector: Detecting and tracking mock objects in unit tests”.
		Submitted to 21st IEEE International Working Conference on 
		Source Code Analysis and Manipulation. Submission date June. 30, 2021}}
\label{chap:introduction}
%======================================================================
\doublespacing

Mock objects~\cite{beck02:_test_driven_devel} are a common idiom in
unit tests for object-oriented systems.  They allow developers to test objects that 
rely on other objects, likely from different components, or that are simply complicated 
to build for testing purposes (e.g. a database).

While mock objects are an invaluable tool for developers, their use
complicates the static analysis and manipulation of test case source code, one of our planned future
research directions. Such static analyses can help IDEs provide better
support to test case writers; enable better static estimation of test coverage
(avoiding mocks); and detect focal methods in test cases.

Ghafari et al discussed the notion of a focal method~\cite{ghafari15:_autom} for a test case---the method
whose behaviour is being tested---and presented a heuristic for determining focal methods.
By definition, the focal method's receiver object cannot be a mock object.
Ruling out mock invocations can thus improve the accuracy of focal method detection and
enable better understanding of a test case's behaviour.

Mock objects are difficult to analyze statically because, at the bytecode level,
a call to a mock object statically resembles a call to the real object (as
intended by the designers of mock libraries).
A naive static analysis attempting to be sound would have to include all of 
the possible behaviours of the actual object (rather than the mock) when analyzing such code. 
Such potential but unrealizable behaviours obscure the true behaviour 
of the test case.

We have designed a static analysis, \textsc{MockDetector}, which identifies
mock objects in test cases. It starts from a list of mock object creation sites
(our analyses include hardcoded APIs for common mocking libraries EasyMock, Mockito, and PowerMock). 
It then propagates mockness
through the test and identifies invocation sites as (possibly) mock.
Given this analysis result, a subsequent analysis
can ask whether a given variable in a test case contains a mock or not, and
whether a given invocation site is a call to a mock object or not. We have
evaluated \textsc{MockDetector} on a suite of 8 benchmarks plus a microbenchmark. 
We have cross-checked results across the two implementations and manually inspected
the results on our microbenchmark, to ensure that the results are as expected.

Taking a broader view, we believe that helper static analyses like \textsc{MockDetector} 
can aid
in the development of more useful static analyses. These analyses can
encode useful domain properties; for instance, in our case, properties
of test cases. By taking a domain-specific approach, analyses can extract
useful facts about programs that would otherwise be difficult to establish.

We make the following contributions in this thesis:
\begin{itemize}
	\item We designed and implemented two variants of a static mock detection algorithm, one as a dataflow analysis implemented imperatively (using Soot) and the other declaratively (using Doop).
	\item We evaluate both the relative ease-of-implementation and precision of the imperative and declarative approaches, both intraprocedurally and interprocedurally (for Doop). % potentially intraprocedural as well
	\item We characterize our benchmark suite (8 open-source benchmarks, 184 kLOC) with respect to their use of mock objects, finding that 1084 out of 6310 unit tests use intraprocedurally-detectable mock objects, and that there are a total of 2095 method invocations on mock objects. %We further identify how powerful an analysis is required to identify mock object use---adding fields and collections adds X mock objects, while interprocedural techniques add Y mock objects.
\end{itemize}
At a higher level, we see the thesis as making both a contribution and a meta-contribution to
problems in source code analysis. The contribution, mock detection, enables more accurate analyses
of test cases, which account for a significant fraction of modern codebases. The meta-contribution,
comparing analysis approaches, will help future researchers decide how to best solve their
source code analysis problems. In brief, the declarative approach allows users to quickly prototype, stating their properties
concisely, while the imperative approach is more amenable to use in program transformation; we return
to this question in Chapter~\ref{chap:discussion}.