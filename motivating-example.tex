%======================================================================
\chapter{Motivating-Examples}
%====================================================================== 

\begin{lstlisting}[basicstyle=\ttfamily, caption={This code snippet illustrates an example from maven-core, where calls to both the focal method \texttt{getToolchainsForType()} and to mock \texttt{session}'s \texttt{getRequest()} method occur in test \textit{testMisconfiguredToolchain()}.},
basicstyle=\scriptsize\ttfamily,language = Java, framesep=4.5mm, escapechar=|,
framexleftmargin=1.0mm, captionpos=b, label=lis:mockCall, morekeywords={@Test}]
@Test
public void testMisconfiguredToolchain()
	throws Exception {
	MavenSession session = mock( MavenSession.class );
	MavenExecutionRequest req = 
	new DefaultMavenExecutionRequest();
	when( session.getRequest() ).thenReturn( req ); |\label{line:mock}|
	
	ToolchainPrivate[] basics =
	toolchainManager.getToolchainsForType("basic", session); |\label{line:real}|
	
	assertEquals( 0, basics.length );
}
\end{lstlisting}


\begin{figure}
	\begin{lstlisting}[basicstyle=\ttfamily,
	basicstyle=\scriptsize\ttfamily,language = Java, framesep=4.5mm, framexleftmargin=1.0mm, captionpos=b, escapechar=|, morekeywords={@Test}]
	//        mock: |\xmark~\,|    mockAPI: |\xmark|
	Object object1 = new Object();
	
	// mock: |\xmark|
	object1.foo();
	
	//        mock: |\cmark|     mockAPI: |\cmark|
	Object object2 = mock(Object.class);
	
	// mock: |\cmark|
	object2.foo();
	\end{lstlisting}
	%    \includegraphics[width=.25\textwidth]{Images/mockInvocationIllustration.png}
	
	\caption{Our static analysis propagates mockness from sources (e.g. \texttt{mock(Object.class}) to invocations.}
	\label{fig:mockMethodIllustration}
	
\end{figure}

\begin{lstlisting}[basicstyle=\ttfamily, caption={Jimple Intermediate Representation for the code in Figure~\ref{fig:mockMethodIllustration}.},
basicstyle=\scriptsize\ttfamily, captionpos=b, label=lis:mockMethodIllustrationIR, escapechar=|, morekeywords={@Test, specialinvoke, virtualinvoke, staticinvoke}]
java.lang.Object $r1, r2;

$r1 = new java.lang.Object; |\label{line:lis3line3}|
specialinvoke $r1.<java.lang.Object: void <init>()>(); |\label{line:lis3line4}|
virtualinvoke $r1.<java.lang.Object: void foo()>(); |\label{line:lis3line5}|
r2 = staticinvoke <org.mockito.Mockito: |\label{line:lis3line6}|
	java.lang.Object mock(java.lang.Class)> |\label{line:lis3line7}|
	(class "Ljava/lang/Object;");
virtualinvoke r2.<java.lang.Object: void foo()>(); |\label{line:lis3line9}|
\end{lstlisting}


\begin{lstlisting}[basicstyle=\ttfamily, caption={Facts about invocation \texttt{r2.foo()} in method \texttt{test}.},
basicstyle=\scriptsize\ttfamily, framesep=4.5mm, framexleftmargin=1.0mm, captionpos=b, label=lis:facts, escapechar=!, morekeywords={@Test}]
isMockInvocation(<Object: void foo()>/test/0, 
<Object: void foo()>, test, _. r2). !\label{line:facts-imi}!
|VirtualMethodInvocation(<Object: void foo()>/test/0, !\label{line:facts-vmi}!
|                        <Object: void foo()>, test).
|VirtualMethodInvocation_Base(<Object: void foo()>/test/0, 
|                                  r2).
|isMockVar(r2). !\label{line:facts-imv}!
|-AssignReturnValue(<Mockito: Object mock(Class)>/test/0, !\label{line:facts-arv}!
|                        r2). 
|-callsMockSource(<Mockito: Object mock(Class)>/test/0). !\label{line:facts-cms}!
|MockSourceMethod(<Mockito: Object mock(Class)>). !\label{line:facts-msm}!
|CallGraphEdge(_, <Mockito: Object mock(Class)>/test/0, _, !\label{line:facts-cge}!
|              <Mockito: Object mock(Class)>). 
\end{lstlisting}


\begin{lstlisting}[basicstyle=\ttfamily, caption={This example illustrates a field array container holding mock objects from \textit{setup()} in \texttt{NodeListIteratorTest.java}.},
basicstyle=\scriptsize\ttfamily,language = Java, framesep=4.5mm, framexleftmargin=1.0mm, captionpos=b, label=lis:container, escapechar=|, morekeywords={@Test}]
// Node array to be filled with mock Node instances
private Node[] nodes;
@Test
protected void setUp() throws Exception {
	// create mock Node Instances and 
	// fill Node[] to be used by test cases
	final Node node1 = createMock(Element.class);
	final Node node2 = createMock(Element.class);
	final Node node3 = createMock(Text.class);
	final Node node4 = createMock(Element.class);
	nodes = new Node[] {node1, node2, node3, node4}; |\label{line:storeMocksInArray}|
	// ...
}
\end{lstlisting}


\begin{figure}
	\begin{lstlisting}[
	basicstyle=\scriptsize\ttfamily,language = Java, framesep=4.5mm, framexleftmargin=1.0mm, captionpos=b, escapechar=|, morekeywords={@Test}]
	//        mock: |\cmark|     mockAPI: |\cmark|
	Object object1 = createMock(Object.class);
	
	// arrayMock: |\cmark| |$\Leftarrow$| array-write    |~~|  mock: |\cmark|
	objects  |~~|           = new Object[]  |~|  { object1 };
	\end{lstlisting}
	
	\caption{Our static analysis also finds array mocks.}
	\label{fig:arrayMockIllustration}
	
\end{figure}

\begin{lstlisting}[basicstyle=\ttfamily, caption={Jimple Intermediate Representation for the array in Figure~\ref{fig:arrayMockIllustration}.},
basicstyle=\scriptsize\ttfamily, framesep=4.5mm, framexleftmargin=1.0mm, captionpos=b, label=lis:arrayIllustrationIR, escapechar=|, morekeywords={@Test, specialinvoke, virtualinvoke, staticinvoke, newarray}]
java.lang.Object r1, $r2;
java.lang.Object[] $r3;

$r2 = staticinvoke <org.easymock.EasyMock: |\label{line:lis4line4}|
java.lang.Object createMock(java.lang.Class)>
(class "java.lang.Object;"); |\label{line:lis4line6}|
r1 = (java.lang.Object) $r2; |\label{line:lis4line7}|
$r3 = newarray (java.lang.Object)[1]; |\label{line:lis4line8}|
$r3[0] = r1;  |\label{line:lis4line9}|
\end{lstlisting}
