%======================================================================
% University of Waterloo Thesis Template for LaTeX 
% Last Updated November, 2020 
% by Stephen Carr, IST Client Services, 
% University of Waterloo, 200 University Ave. W., Waterloo, Ontario, Canada
% FOR ASSISTANCE, please send mail to request@uwaterloo.ca

% DISCLAIMER
% To the best of our knowledge, this template satisfies the current uWaterloo thesis requirements.
% However, it is your responsibility to assure that you have met all requirements of the University and your particular department.

% Many thanks for the feedback from many graduates who assisted the development of this template.
% Also note that there are explanatory comments and tips throughout this template.
%======================================================================
% Some important notes on using this template and making it your own...

% The University of Waterloo has required electronic thesis submission since October 2006. 
% See the uWaterloo thesis regulations at
% https://uwaterloo.ca/graduate-studies/thesis.
% This thesis template is geared towards generating a PDF version optimized for viewing on an electronic display, including hyperlinks within the PDF.

% DON'T FORGET TO ADD YOUR OWN NAME AND TITLE in the "hyperref" package configuration below. 
% THIS INFORMATION GETS EMBEDDED IN THE PDF FINAL PDF DOCUMENT.
% You can view the information if you view properties of the PDF document.

% Many faculties/departments also require one or more printed copies. 
% This template attempts to satisfy both types of output. 
% See additional notes below.
% It is based on the standard "book" document class which provides all necessary sectioning structures and allows multi-part theses.

% If you are using this template in Overleaf (cloud-based collaboration service), then it is automatically processed and previewed for you as you edit.

% For people who prefer to install their own LaTeX distributions on their own computers, and process the source files manually, the following notes provide the sequence of tasks:
 
% E.g. to process a thesis called "mythesis.tex" based on this template, run:

% pdflatex mythesis	-- first pass of the pdflatex processor
% bibtex mythesis	-- generates bibliography from .bib data file(s)
% makeindex         -- should be run only if an index is used 
% pdflatex mythesis	-- fixes numbering in cross-references, bibliographic references, glossaries, index, etc.
% pdflatex mythesis	-- it takes a couple of passes to completely process all cross-references

% If you use the recommended LaTeX editor, Texmaker, you would open the mythesis.tex file, then click the PDFLaTeX button. Then run BibTeX (under the Tools menu).
% Then click the PDFLaTeX button two more times. 
% If you have an index as well,you'll need to run MakeIndex from the Tools menu as well, before running pdflatex
% the last two times.

% N.B. The "pdftex" program allows graphics in the following formats to be included with the "\includegraphics" command: PNG, PDF, JPEG, TIFF
% Tip: Generate your figures and photos in the size you want them to appear in your thesis, rather than scaling them with \includegraphics options.
% Tip: Any drawings you do should be in scalable vector graphic formats: SVG, PNG, WMF, EPS and then converted to PNG or PDF, so they are scalable in the final PDF as well.
% Tip: Photographs should be cropped and compressed so as not to be too large.

% To create a PDF output that is optimized for double-sided printing: 
% 1) comment-out the \documentclass statement in the preamble below, and un-comment the second \documentclass line.
% 2) change the value assigned below to the boolean variable "PrintVersion" from " false" to "true".

%======================================================================
%   D O C U M E N T   P R E A M B L E
% Specify the document class, default style attributes, and page dimensions, etc.
% For hyperlinked PDF, suitable for viewing on a computer, use this:
\documentclass[letterpaper,12pt,titlepage,oneside,final]{book}
 
% For PDF, suitable for double-sided printing, change the PrintVersion variable below to "true" and use this \documentclass line instead of the one above:
%\documentclass[letterpaper,12pt,titlepage,openright,twoside,final]{book}

% Some LaTeX commands I define for my own nomenclature.
% If you have to, it's easier to make changes to nomenclature once here than in a million places throughout your thesis!
\newcommand{\package}[1]{\textbf{#1}} % package names in bold text
\newcommand{\cmmd}[1]{\textbackslash\texttt{#1}} % command name in tt font 
\newcommand{\href}[1]{#1} % does nothing, but defines the command so the print-optimized version will ignore \href tags (redefined by hyperref pkg).
%\newcommand{\texorpdfstring}[2]{#1} % does nothing, but defines the command
% Anything defined here may be redefined by packages added below...

\usepackage{verbatim}
\usepackage{csvsimple}
\usepackage[utf8]{inputenc}

\newcommand{\ra}[1]{\renewcommand{\arraystretch}{#1}}
\usepackage{array, lscape, booktabs, makecell} % For formal tables
\usepackage{xcolor}
\usepackage{listings}
\usepackage{pifont}% http://ctan.org/pkg/pifont
\newcommand{\cmark}{\ding{51}}%
\newcommand{\xmark}{\ding{55}}%

\definecolor{vdarkgreen}{rgb}{0,0.3,0}
\lstset{
	language=Java,
	columns=flexible,
	numbers=left,
	numbersep=2pt,
	stepnumber=1,
	keywordstyle=\color{blue},
	tabsize=1,
	keywordstyle=\color{blue}\bfseries,
	commentstyle=\color{vdarkgreen},
	stringstyle=\ttfamily\color{red!50!brown},
	showstringspaces=false}

% This package allows if-then-else control structures.
\usepackage{ifthen}
\newboolean{PrintVersion}
\setboolean{PrintVersion}{false}
% CHANGE THIS VALUE TO "true" as necessary, to improve printed results for hard copies by overriding some options of the hyperref package, called below.


%\usepackage{nomencl} % For a nomenclature (optional; available from ctan.org)
\usepackage{amsmath, amsthm, amssymb, amsfonts} % Lots of math symbols and environments
\usepackage[pdftex]{graphicx} % For including graphics N.B. pdftex graphics driver 
% Hyperlinks make it very easy to navigate an electronic document.
% In addition, this is where you should specify the thesis title and author as they appear in the properties of the PDF document.
% Use the "hyperref" package 
% N.B. HYPERREF MUST BE THE LAST PACKAGE LOADED; ADD ADDITIONAL PKGS ABOVE
\usepackage[pdftex,pagebackref=false]{hyperref} % with basic options
%\usepackage[pdftex,pagebackref=true]{hyperref}
		% N.B. pagebackref=true provides links back from the References to the body text. This can cause trouble for printing.
\hypersetup{
    plainpages=false,       % needed if Roman numbers in frontpages
    unicode=false,          % non-Latin characters in Acrobat’s bookmarks
    pdftoolbar=true,        % show Acrobat’s toolbar?
    pdfmenubar=true,        % show Acrobat’s menu?
    pdffitwindow=false,     % window fit to page when opened
    pdfstartview={FitH},    % fits the width of the page to the window
%    pdftitle={uWaterloo\ LaTeX\ Thesis\ Template},    % title: CHANGE THIS TEXT!
%    pdfauthor={Author},    % author: CHANGE THIS TEXT! and uncomment this line
%    pdfsubject={Subject},  % subject: CHANGE THIS TEXT! and uncomment this line
%    pdfkeywords={keyword1} {key2} {key3}, % list of keywords, and uncomment this line if desired
    pdfnewwindow=true,      % links in new window
    colorlinks=true,        % false: boxed links; true: colored links
    linkcolor=blue,         % color of internal links
    citecolor=green,        % color of links to bibliography
    filecolor=magenta,      % color of file links
    urlcolor=cyan           % color of external links
}

\usepackage{tikz}
\usetikzlibrary{arrows,automata,shapes,positioning}
\tikzset{
	normal/.style = {
		rectangle split, 
		rectangle split parts=2, 
		very thick, draw=black, 
	}
}

\ifthenelse{\boolean{PrintVersion}}{   % for improved print quality, change some hyperref options
\hypersetup{	% override some previously defined hyperref options
%    colorlinks,%
    citecolor=black,%
    filecolor=black,%
    linkcolor=black,%
    urlcolor=black}
}{} % end of ifthenelse (no else)

\usepackage[automake,toc,abbreviations]{glossaries-extra} % Exception to the rule of hyperref being the last add-on package
% If glossaries-extra is not in your LaTeX distribution, get it from CTAN (http://ctan.org/pkg/glossaries-extra), 
% although it's supposed to be in both the TeX Live and MikTeX distributions. There are also documentation and 
% installation instructions there.

% Setting up the page margins...
% uWaterloo thesis requirements specify a minimum of 1 inch (72pt) margin at the
% top, bottom, and outside page edges and a 1.125 in. (81pt) gutter margin (on binding side). 
% While this is not an issue for electronic viewing, a PDF may be printed, and so we have the same page layout for both printed and electronic versions, we leave the gutter margin in.
% Set margins to minimum permitted by uWaterloo thesis regulations:
\setlength{\marginparwidth}{0pt} % width of margin notes
% N.B. If margin notes are used, you must adjust \textwidth, \marginparwidth
% and \marginparsep so that the space left between the margin notes and page
% edge is less than 15 mm (0.6 in.)
\setlength{\marginparsep}{0pt} % width of space between body text and margin notes
\setlength{\evensidemargin}{0.125in} % Adds 1/8 in. to binding side of all 
% even-numbered pages when the "twoside" printing option is selected
\setlength{\oddsidemargin}{0.125in} % Adds 1/8 in. to the left of all pages when "oneside" printing is selected, and to the left of all odd-numbered pages when "twoside" printing is selected
\setlength{\textwidth}{6.375in} % assuming US letter paper (8.5 in. x 11 in.) and side margins as above
\raggedbottom

% The following statement specifies the amount of space between paragraphs. Other reasonable specifications are \bigskipamount and \smallskipamount.
\setlength{\parskip}{\medskipamount}

% The following statement controls the line spacing.  
% The default spacing corresponds to good typographic conventions and only slight changes (e.g., perhaps "1.2"), if any, should be made.
\renewcommand{\baselinestretch}{1} % this is the default line space setting

% By default, each chapter will start on a recto (right-hand side) page.
% We also force each section of the front pages to start on a recto page by inserting \cleardoublepage commands.
% In many cases, this will require that the verso (left-hand) page be blank, and while it should be counted, a page number should not be printed.
% The following statements ensure a page number is not printed on an otherwise blank verso page.
\let\origdoublepage\cleardoublepage
\newcommand{\clearemptydoublepage}{%
  \clearpage{\pagestyle{empty}\origdoublepage}}
\let\cleardoublepage\clearemptydoublepage

% Define Glossary terms (This is properly done here, in the preamble and could also be \input{} from a separate file...)
% Main glossary entries -- definitions of relevant terminology
\newglossaryentry{computer}
{
name=computer,
description={A programmable machine that receives input data,
               stores and manipulates the data, and provides
               formatted output}
}

% Nomenclature glossary entries -- New definitions, or unusual terminology
\newglossary*{nomenclature}{Nomenclature}
\newglossaryentry{dingledorf}
{
type=nomenclature,
name=dingledorf,
description={A person of supposed average intelligence who makes incredibly brainless misjudgments}
}

% List of Abbreviations (abbreviations type is built in to the glossaries-extra package)
\newabbreviation{aaaaz}{AAAAZ}{American Association of Amateur Astronomers and Zoologists}

% List of Symbols
\newglossary*{symbols}{List of Symbols}
\newglossaryentry{rvec}
{
name={$\mathbf{v}$},
sort={label},
type=symbols,
description={Random vector: a location in n-dimensional Cartesian space, where each dimensional component is determined by a random process}
}
\makeglossaries

\usepackage{setspace}

%======================================================================
%   L O G I C A L    D O C U M E N T
% The logical document contains the main content of your thesis.
% Being a large document, it is a good idea to divide your thesis into several files, each one containing one chapter or other significant chunk of content, so you can easily shuffle things around later if desired.
%======================================================================
\begin{document}

%----------------------------------------------------------------------
% FRONT MATERIAL
% title page,declaration, borrowers' page, abstract, acknowledgements,
% dedication, table of contents, list of tables, list of figures, nomenclature, etc.
%----------------------------------------------------------------------
% T I T L E   P A G E
% -------------------
% Last updated October 23, 2020, by Stephen Carr, IST-Client Services
% The title page is counted as page `i' but we need to suppress the
% page number. Also, we don't want any headers or footers.
\pagestyle{empty}
\pagenumbering{roman}

% The contents of the title page are specified in the "titlepage"
% environment.
\begin{titlepage}
        \begin{center}
        \vspace*{1.0cm}

        \Huge
        {\bf MockDetector: Detecting and tracking mock objects in unit tests }

        \vspace*{1.0cm}

        \normalsize
        by \\

        \vspace*{1.0cm}

        \Large
        Qian Liang \\

        \vspace*{3.0cm}

        \normalsize
        A thesis \\
        presented to the University of Waterloo \\ 
        in fulfillment of the \\
        thesis requirement for the degree of \\
        Master of Applied Science \\
        in \\
        Electrical and Computer Engineering \\

        \vspace*{2.0cm}

        Waterloo, Ontario, Canada, 2021 \\

        \vspace*{1.0cm}

        \copyright\ Qian Liang 2021 \\
        \end{center}
\end{titlepage}

% The rest of the front pages should contain no headers and be numbered using Roman numerals starting with `ii'
\pagestyle{plain}
\setcounter{page}{2}

\cleardoublepage % Ends the current page and causes all figures and tables that have so far appeared in the input to be printed.
% In a two-sided printing style, it also makes the next page a right-hand (odd-numbered) page, producing a blank page if necessary.

 
% E X A M I N I N G   C O M M I T T E E (Required for Ph.D. theses only)
% Remove or comment out the lines below to remove this page
\begin{center}\textbf{Examining Committee Membership}\end{center}
  \noindent
The following served on the Examining Committee for this thesis. The decision of the Examining Committee is by majority vote.
  \bigskip
  
  \noindent
\begin{tabbing}
Internal-External Member: \=  \kill % using longest text to define tab length
Supervisor(s): \> Patrick Lam \\
\> Associate Professor \\
\> University of Waterloo \\
\end{tabbing}
  \bigskip
  
  \noindent
  \begin{tabbing}
Internal-External Member: \=  \kill % using longest text to define tab length
Internal Member: \> Derek Rayside \\
\> Associate Professor \\
\> University of Waterloo \\
\end{tabbing}
  \bigskip
  
  \noindent
\begin{tabbing}
Internal-External Member: \=  \kill % using longest text to define tab length
Internal Member: \> Gregor Richards \\
\> Lecturer \\
\> University of Waterloo \\
\end{tabbing}
  \bigskip

\cleardoublepage

% D E C L A R A T I O N   P A G E
% -------------------------------
 \begin{center}\textbf{Author's Declaration}\end{center}
  
 \noindent
This thesis consists of material all of which I authored or co-authored: see Statement of Contributions included in the thesis. This is a true copy of the thesis, including any required final revisions, as accepted by my examiners.
  \bigskip
  
  \noindent
I understand that my thesis may be made electronically available to the public.

\cleardoublepage

% S T A T E M E N T   O F   C O N T R I B U T I O N S
% ---------------------------------------------------
\begin{center}\textbf{Statement of Contributions}\end{center}

This thesis consists of all chapters written for conference research paper submission, with minor word changes and styling updates.

Qian Liang was the sole author for Chapters~\ref{chap:motivating-example},~\ref{chap:evaluation}, and Sections~\ref{sec:soot},~\ref{sec:common}, which were written under the supervision of Dr. Patrick Lam. I was responsible for developing the imperative Soot implementation for mock analysis, carrying out data collection and analysis from both imperative Soot and declarative Doop's implementations.

Qian Liang and Dr. Patrick Lam were the co-authors for Chapters~\ref{chap:introduction},~\ref{chap:related} and~\ref{chap:discussion}. 

Dr. Patrick Lam was the sole author for Section~\ref{sec:dec-doop}.

\cleardoublepage

% A B S T R A C T
% ---------------

\begin{center}\textbf{Abstract}\end{center}

% Used the abstract from research paper submission for now. Will modify the text.
Unit testing is a widely used tool in modern software development processes. A well-known issue in writing tests is handling dependencies: creating usable objects for dependencies is often complicated. Developers must therefore often introduce mock objects to stand in for dependencies during testing. 

Test suites are an increasingly-important component of the source code of a software system. We believe that the static analysis of test suites, alongside the systems under test, can enable developers to better characterize the behaviours of existing test suites, thus guiding further test suite analysis and manipulation. However, because mock objects are created using reflection, they confound existing static analysis techniques. At present, it is impossible to statically distinguish methods invoked on mock objects from methods invoked on real objects. Static analysis tools therefore currently cannot determine which dependencies' methods are actually tested, versus mock methods being called.

In this thesis, we introduce MockDetector, a technique to identify mock objects and track method invocations on mock objects. We first built a Soot-based imperative dataflow analysis implementation of MockDetector. Then, to quickly prototype new analysis features and to explore declarative program analysis, we created a Doop-based declarative analysis, added features to it, and ported them back to the Soot-based analysis. Both analyses handle common Java mock libraries' APIs for creating mock objects and propagate this information through test cases. Following our observations of tests in the wild, we have added special-case support for arrays and collections holding mock objects. On our suite of 8 open-source benchmarks, our imperative dataflow analysis approach reported 2,095 invocations on mock objects, whereas our declarative dataflow approach reported 2,130 invocations on mock objects, out of a total number of 63,017 method invocations in test suites; across benchmarks, mock invocations accounted for a range from 0.086\% to 16.4\% of the total invocations. Removing confounding mock invocations from consideration as focal methods can improve the precision of focal method analysis, a key prerequisite to further analysis of test cases. %Both implementations have reported the same number of intra-procedural mock invocations on 4 out of the 8 open source benchmarks analyzed. 

% results part to be updated

\cleardoublepage

% A C K N O W L E D G E M E N T S
% -------------------------------

\begin{center}\textbf{Acknowledgements}\end{center}

I would like to thank to my advisor, Professor Patrick Lam, without whom the thesis
would not have been possible. I would also like to thank to the readers of the thesis,
Professor Gregor Richards and Professor Derek Rayside.
\cleardoublepage

% D E D I C A T I O N
% -------------------

\begin{center}\textbf{Dedication}\end{center}

I dedicate the thesis to my family, who have always supported me during my study and life at University of Waterloo.
\cleardoublepage

% T A B L E   O F   C O N T E N T S
% ---------------------------------
\renewcommand\contentsname{Table of Contents}
\tableofcontents
\cleardoublepage
\phantomsection    % allows hyperref to link to the correct page

% L I S T   O F   F I G U R E S
% -----------------------------
\addcontentsline{toc}{chapter}{List of Figures}
\listoffigures
\cleardoublepage
\phantomsection		% allows hyperref to link to the correct page

% L I S T   O F   T A B L E S
% ---------------------------
\addcontentsline{toc}{chapter}{List of Tables}
\listoftables
\cleardoublepage
\phantomsection		% allows hyperref to link to the correct page

% Change page numbering back to Arabic numerals
\pagenumbering{arabic}

 

%----------------------------------------------------------------------
% MAIN BODY
% We suggest using a separate file for each chapter of your thesis.
% Start each chapter file with the \chapter command.
% Only use \documentclass or \begin{document} and \end{document} commands in this master document.
% Tip: Putting each sentence on a new line is a way to simplify later editing.
%----------------------------------------------------------------------
%======================================================================
\chapter{Introduction}
\label{chap:introduction}
%======================================================================
\doublespacing

Mock objects~\cite{beck02:_test_driven_devel} are a common idiom in
unit tests for object-oriented systems.  They allow developers to test objects that 
rely on other objects, likely from different components, or that are simply complicated 
to build for testing purposes (e.g. a database).

While mock objects are an invaluable tool for developers, their use
complicates the static analysis and manipulation of test case source code, one of our planned future
research directions. Such static analyses can help IDEs provide better
support to test case writers; enable better static estimation of test coverage
(avoiding mocks); and detect focal methods in test cases.

Ghafari et al discussed the notion of a focal method~\cite{ghafari15:_autom} for a test case---the method
whose behaviour is being tested---and presented a heuristic for determining focal methods.
By definition, the focal method's receiver object cannot be a mock object.
Ruling out mock invocations can thus improve the accuracy of focal method detection and
enable better understanding of a test case's behaviour.

Mock objects are difficult to analyze statically because, at the bytecode level,
a call to a mock object statically resembles a call to the real object (as
intended by the designers of mock libraries).
A naive static analysis attempting to be sound would have to include all of 
the possible behaviours of the actual object (rather than the mock) when analyzing such code. 
Such potential but unrealizable behaviours obscure the true behaviour 
of the test case.

We have designed a static analysis, \textsc{MockDetector}, which identifies
mock objects in test cases. It starts from a list of mock object creation sites
(our analyses include hardcoded APIs for common mocking libraries EasyMock, Mockito, and PowerMock). 
It then propagates mockness
through the test and identifies invocation sites as (possibly) mock.
Given this analysis result, a subsequent analysis
can ask whether a given variable in a test case contains a mock or not, and
whether a given invocation site is a call to a mock object or not. We have
evaluated \textsc{MockDetector} on a suite of 8 benchmarks plus a microbenchmark. 
We have cross-checked results across the two implementations and manually inspected
the results on our microbenchmark, to ensure that the results are as expected.

Taking a broader view, we believe that helper static analyses like \textsc{MockDetector} 
can aid
in the development of more useful static analyses. These analyses can
encode useful domain properties; for instance, in our case, properties
of test cases. By taking a domain-specific approach, analyses can extract
useful facts about programs that would otherwise be difficult to establish.

We make the following contributions in this thesis:
\begin{itemize}
	\item We designed and implemented two variants of a static mock detection algorithm, one as a dataflow analysis implemented imperatively (using Soot) and the other declaratively (using Doop).
	\item We evaluate both the relative ease-of-implementation and precision of the imperative and declarative approaches, both intraprocedurally and interprocedurally (for Doop). % potentially intraprocedural as well
	\item We characterize our benchmark suite (8 open-source benchmarks, 184 kLOC) with respect to their use of mock objects, finding that 1084 out of 6310 unit tests use intraprocedurally-detectable mock objects, and that there are a total of 2095 method invocations on mock objects. %We further identify how powerful an analysis is required to identify mock object use---adding fields and collections adds X mock objects, while interprocedural techniques add Y mock objects.
\end{itemize}
At a higher level, we see the thesis as making both a contribution and a meta-contribution to
problems in source code analysis. The contribution, mock detection, enables more accurate analyses
of test cases, which account for a significant fraction of modern codebases. The meta-contribution,
comparing analysis approaches, will help future researchers decide how to best solve their
source code analysis problems. In brief, the declarative approach allows users to quickly prototype, stating their properties
concisely, while the imperative approach is more amenable to use in program transformation; we return
to this question in Chapter~\ref{chap:discussion}.

%======================================================================
\chapter{Background}
\label{chap:background}
%====================================================================== 

Static analysis techniques are an important tool for code analysis, especially when analyzing benchmarks with large code bases. Running an applicaion does not always expose bugs (for instance, due to missing branch coverage), where static analysis tools are more competent. A good static analysis tool can help developers to find bugs that are hard to spot manually (e.g., array out of bounds exception), which reduces time and costs involved in fixing bugs. 

\section{Dataflow Analysis}

Dataflow analysis is normally performed on top of a program's control flow graph (CFG). A CFG is a directed graph with each node stands for a statement in the program (the statement could be an assignment statement, an if statement, etc.), and the set of directed edges represents the overall control flow of the program.

Dataflow analysis, meanwhile, is a fixed point algorithm that computes and gathers all the facts at each program point (i.e., each node in the CFG). The facts would usually be a mapping between program variables and the abstractions specifically defined to solve the problem at hand. 

For each dataflow problem, researchers or developers usually make two decisions before implementations. First, they would decide if the problem should be categorized as a forward or backward dataflow problem. For a forward dataflow analysis, the facts propagate along the direction of the control flow. Determining whether expressions are available at each program point is a type of forward dataflow problem. On the other hand, the facts propagate in the opposite direction from the control flow in a backward dataflow analysis, and most importantly, we need to access the future usage of the variables. Analyzing the liveness of variables is a known backward dataflow problem. (Few dataflow problems may require bidirectional flows.)

Researchers or developers also need to decide on if it is a may or must dataflow problem. The core difference between the two types of problems is how they handle the facts at all the join points in the program, where multiple branches meet. A may dataflow analysis keeps facts that hold true on any joined path. "Reaching Definitions" is a may analysis problem. It checks if a definition (or assignment) of a variable reaches a specific program point. On the other hand, must dataflow analysis only keeps facts that hold true from all the branches. Determining available expressions is also a must analysis problem.

Dataflow analysis is an imperative approach when solving a problem. It focuses on the ``HOW" part of the solution, providing a set of commands or operations to tackle the problem.

\subsection{Soot}

Among all the dataflow analysis tools, Soot~\cite{Vallee-Rai:1999:SJB:781995.782008} is a representative, IFDS-based Java optimization framework (where IFDS stands for Interprocedural, finite, distributive, subset problems), which provides a total of four types of intermediate representations to analyze Java bytecode. 

An intermediate representation (IR) is an abstract language designed for machines with no specifications. A good IR is independent of the source and target languages, and thus convenient to translate into code for the retargetable architecture.

For our project, we use Jimple (Java's simple), which is a stackless, typed 3-address IR in the CFG.

\section{Declarative Analysis}

While an imperative approach focuses on the ``HOW" component of a solution, declarative analysis focuses on the ``WHAT" part during the implementation. It gives out a set of orders for the framework to achieve.

It worths noting that a declarative approach normally comes with an underlying imperative layer, where some type of tool handles the imperative processes for the developer.

\subsection{Doop}

We choose Doop among many declarative frameworks for our project. It comes with a batch of base ``analyses expressed in the form of Datalog rules"~\cite{doop-repo}. The current version uses Soufflé\footnote{\url{https://souffle-lang.github.io/docs.html}} as the datalog engines for analyses rules.

Since Doop is a declarative framework, its underlying imperative layer is the auto-generated input facts from Soot framework, which consequently imported to Doop's database. These input facts are then processed by the implemented datalog rules (on top of the selected Soufflé base analysis). Doop also computes fixed points like any dataflow analysis tools.


\section{Mock Objects} 

Mock objects are commonly used in unit test cases. They substitute for the real objects that are normally hard to create (e.g., a database), or slow to process (e.g., a connection). It is noteworthy that mock objects are never being tested in test cases. Their purpose are to stand in for the dependencies, assisting for the behavioural test of the real object. For this to happen, the mock objects must at least mimic the behaviour at interfaces connecting to the real object under test.

Imagine you want to test the behaviour of a detonator. It is infeasible to always test the detonator's behaviour with a real bomb, also sometimes there is no bomb available. So in this scenario, you build a mock bomb. The mock bomb does not do anything other than checks whether it has received instructions to explode. Note that in the whole process, you are testing the detonator's behaviour, not the mock bomb's behaviour.\footnote{The idea of this example is sparked from \url{https://stackoverflow.com/questions/28783722/when-using-mokito-what-is-the-difference-between-the-actual-object-and-the-mock?noredirect=1&lq=1}.}

The current static analysis tools, however, could not differentiate a mock object and a real object, because the two objects' IR are essentially of the same type. Section~\ref{sec:toy-example} runs a toy example and explains why the current tools are incompetent in locating mock objects.

For our project, we consider for Java mock source methods from three mocking frameworks: EasyMock\footnote{\url{https://easymock.org/}}, Mockito\footnote{\url{https://site.mockito.org/}}, and PowerMock\footnote{\url{https://github.com/powermock/powermock}}. According to a prior study~\cite{mostafa14:_empirical_study_mock_frameworks}, EasyMock and Mockito are used in about 90\% of the 5,000 randomly sampled projects. We then added a third mocking framework by our own choice. Thus, we believe our analysis results should be applicable to most of the Java benchmarks. 

%======================================================================
\chapter{Motivation}
\label{chap:motivation}
%====================================================================== 

In this chapter, we illustrate how \textsc{MockDetector} finds variables containing mock objects and mock invocations in unit tests---invocations with mock objects as receiver objects. Our tool identifies invocations on variables which have been assigned an object flowing from a mock creation site either using a forward dataflow may-analysis (Soot-based analysis) or by solving specified declarative constraints (Doop-based analysis). Before diving into the current project, let us take a step back and talk about what led us to the research on detecting mock objects and tracking mock invocations.

\section{Preliminary Research}

\begin{figure}[h]
	\centering
	\tikzstyle{block} = [rectangle, draw, 
text width=3em, text centered, rounded corners, minimum height=2em]

\begin{tikzpicture}[>=stealth',shorten >=1pt,auto,node distance=2cm,
semithick,initial text=]

\node[normal,block,text width=7em]   (A)              {\textit{Assembled-Chronology}};

\node[normal, xshift=-8em, yshift=-3em, text width=6em, text centered,rounded corners] (B) [below left of=A] {\nodepart{one} \textit{Buddhist-Chronology} 
\nodepart{two} \textit{withZone()}:\\TESTED $\checkmark$ };

\node[normal, xshift=0.5em, yshift=-3em, text width=6em, text centered,rounded corners] (D) [below left of=A] {\nodepart{one} \textit{GJ-Chronology} 
	\nodepart{two} \textit{withZone()}:\\TESTED $\checkmark$ };

\node[normal,xshift=1em,text width=1em,draw=none]           (G) [below of=A] {\ldots};

\node[normal, xshift=10em, yshift=-3em, text width=6.3em, text centered,rounded corners] (S) [below left of=A] {\nodepart{one} \textit{Strict-Chronology} 
	\nodepart{two} \textit{withZone()}:\\UNTESTED~$\times$};


\draw (B) -> (A);
\draw (G) -> (A);
\draw (D) -> (A);
\draw (S) -> (A); 

\end{tikzpicture} 
	\caption[Caption for SiblingClass untested method.]{joda-time contains superclass \textit{AssembledChronology} and subclasses \textit{BuddhistChronology}, \textit{GJChronology}, \textit{StrictChronology}, and others. Method \textit{withZone()} is tested in \textit{Buddhist-} and \textit{GJChronology} but not \textit{StrictChronology}.\footnotemark}
	\label{fig:hierarchyView}
\end{figure}

\footnotetext{The content of this figure have been incorporated within a NIER paper published by IEEE in 2020 IEEE International Conference on Software Maintenance and Evolution (ICSME), available online: https://ieeexplore.ieee.org/xpl/conhome/9240597/proceeding[doi: 10.1109/ICSME46990.2020.00075]. Qian Liang and Patrick Lam, "SiblingClassTestDetector: Finding Untested Sibling Functions"}

Prior to the development for \textsc{MockDetector}, we were working on a project to detect untested functions that have analogous implementations in sibling classes (they share a common superclass), where at least one of the related implementations are tested. The overall goal of that project is to reduce untested code. Though testing could not guarantee desired program behaviour, developers certainly know nothing about any untested code. Since the sibling methods share the same specification, it is likely that a unit test case covering for one sibling class's implementation may also work for the untested after small modifications, potentially increasing the statement coverage and consequently having a better chance to gain behavioural insight of the benchmark.

Figure~\ref{fig:hierarchyView} illustrates such an example from an open-source benchmark, joda-time (version 2.10.5). The abstract class \textit{AssembledChronology} inherits the specification of method \textit{withZone()} from its parent class, which is not shown in the Figure. \textit{AssembledChronology}'s subclasses \textit{BuddhistChronology}, \textit{GJChronology}, \textit{StrictChronology}, and many others are at the same hierarchy level, which are defined as sibling classes. These sibling classes all have an implementation of \textit{withZone()}; however, the \textit{withZone()} implementations in \textit{BuddhistChronology} and \textit{GJChronology} are tested, whereas the implementation in \textit{StrictChronology} is not.

As the research progressed, we encountered the test case in Listing~\ref{lis:siblingMethodCall}, and realized that it would be a necessary step to remove method invocations on mock objects from the call graph generated by existing static analysis frameworks, as otherwise we may mistakenly treat such test case as the one covering for the tested sibling method, since the existing static analyses tools could not distinguish method invocations on mock objects from method invocations on real objects. 

\begin{lstlisting}[basicstyle=\ttfamily, caption={This code snippet illustrates an example from commons-collections4, where the method \textit{addAll()} invoked on the mock object \texttt{c} could be mistreated as a focal method being covered by existing static analysis frameworks.},
basicstyle=\ttfamily,language = Java, framesep=4.5mm, escapechar=|,
framexleftmargin=1.0mm, captionpos=b, label=lis:siblingMethodCall, morekeywords={@Test}]

@Test
public void addAllForIterable() {
final Collection<Integer> inputCollection = createMock(Collection.class);
...
final Collection<Number> c = createMock(Collection.class);
...
expect(c.addAll(inputCollection)).andReturn(false);
}
\end{lstlisting}

\section{Running Example: Detecting Mock Objects and Mock Invocations}

To motivate our work, consider Listing~\ref{lis:mockCall}, which presents a unit test case from the Maven project. Line~\ref{line:mock} calls \textit{getRequest()}, invoking it on the mock object \texttt{session}. Line~\ref{line:real} then calls \textit{getToolchainsForType()}, which happens to be the focal method whose behaviour is being tested in this test case. At the bytecode level, the two method invocations are indistinguishable with respect to mockness; to our knowledge, current static analysis tools cannot easily tell the difference between the method invocation on a mock object on line~\ref{line:mock} and the method invocation on a real object on line~\ref{line:real}. Given mockness information, an IDE could provide better suggestions. The uncertainty about mockness would confound a naive static analysis that attempts to identify focal methods. For instance, Ghafari et al's heuristic~\cite{ghafari15:_autom} would fail on this test, as it returns the last mutator method in the object under test, and the focal method here is an accessor.

\begin{lstlisting}[basicstyle=\ttfamily, caption={This code snippet illustrates an example from maven-core, where calls to both the focal method \texttt{getToolchainsForType()} and to mock \texttt{session}'s \texttt{getRequest()} method occur in the test \textit{testMisconfiguredToolchain()}.},
basicstyle=\ttfamily,language = Java, framesep=4.5mm, escapechar=|,
framexleftmargin=1.0mm, captionpos=b, label=lis:mockCall, morekeywords={@Test}]
@Test
public void testMisconfiguredToolchain() throws Exception {
	MavenSession session = mock( MavenSession.class );
	MavenExecutionRequest req = new DefaultMavenExecutionRequest();
	when( session.getRequest() ).thenReturn( req ); |\label{line:mock}|

	ToolchainPrivate[] basics =
			toolchainManager.getToolchainsForType("basic", session); |\label{line:real}|

	assertEquals( 0, basics.length );
}
\end{lstlisting}

\section{Basic Dataflow Analysis} 

The dataflow-analysis-based \textsc{MockDetector} implementation maintains an abstraction mapping values (local variables or field references) in Soot's Jimple intermediate representation (IR)~\cite{Vallee-Rai:1999:SJB:781995.782008} to \texttt{MockStatus}, which holds three bits monitoring each value's status being a mock, a mock-containing array, and a mock-containing collection, respectively. The declarative version, which we do not explain in this section, maintains relations that track the same bits.

Figure~\ref{fig:mockMethodIllustration} shows how our dataflow analysis works, and Listing~\ref{lis:mockMethodIllustrationIR} below shows the Jimple IR of the code in Figure~\ref{fig:mockMethodIllustration}. At the top of the Jimple IR, we begin with an empty abstraction (no mapping for any values, equivalent to all bits false for each value) before line~\ref{line:lis3line3}. For the creation of \texttt{\$r1} on line~\ref{line:lis3line3} and~\ref{line:lis3line4}, since the call to the no-arg \texttt{<init>} constructor is not one of our hardcoded mock APIs, our analysis does not declare \texttt{\$r1} to be a mock object. In practice, our abstraction simply does not create an explicit binding for \texttt{\$r1}, instead leaving the mapping empty as it was prior to line~\ref{line:lis3line3}; but it would be equivalent to create a new \texttt{MockStatus} with all bits false and bind it to \texttt{\$r1}. Thus, we may conclude that the invocation \texttt{object1.foo()} on line 5 in Figure~\ref{fig:mockMethodIllustration} is not known to be a mock invocation. Tying back to our focal methods application, we would not exclude the call to \texttt{foo()} from being a possible focal method.

\begin{figure}
	\begin{lstlisting}[basicstyle=\ttfamily,
	basicstyle=\ttfamily,language = Java, framesep=4.5mm, framexleftmargin=1.0mm, captionpos=b, escapechar=|, morekeywords={@Test}]
	//        mock: |\xmark~\,|    mockAPI: |\xmark|
	Object object1 = new Object();
	
	// mock: |\xmark|
	object1.foo();
	
	//        mock: |\cmark|     mockAPI: |\cmark|
	Object object2 = mock(Object.class);
	
	// mock: |\cmark|
	object2.foo();
	\end{lstlisting}
	%    \includegraphics[width=.25\textwidth]{Images/mockInvocationIllustration.png}
	
	\caption{Our static analysis propagates mockness from sources (e.g. \texttt{mock(Object.class}) to invocations.}
	\label{fig:mockMethodIllustration}
	
\end{figure}

On the other hand, our imperative analysis sees the mock creation API \\ \texttt{<org.mockito.Mockito: java.lang.Object mock(java.lang.Class)>} on line~\ref{line:lis3line6} and~\ref{line:lis3line7} in the Jimple IR. It thus adds a mapping from local variable \texttt{r2} to a new \texttt{MockStatus} with the mock bit set to true. When the analysis reaches line~\ref{line:lis3line9}, because \texttt{r2} has a mapping in the abstraction with the mock bit being set, \textsc{MockDetector} will deduce that the call on line~\ref{line:lis3line7} is a mock invocation. This implies that the call to method \textit{foo()} on line 11 in Figure~\ref{fig:mockMethodIllustration} cannot be a focal method.


\begin{lstlisting}[basicstyle=\ttfamily, caption={Jimple Intermediate Representation for the code in Figure~\ref{fig:mockMethodIllustration}.},
basicstyle=\ttfamily, captionpos=b, label=lis:mockMethodIllustrationIR, escapechar=|, morekeywords={@Test, specialinvoke, virtualinvoke, staticinvoke}]
java.lang.Object $r1, r2;

$r1 = new java.lang.Object; |\label{line:lis3line3}|
specialinvoke $r1.<java.lang.Object: void <init>()>(); |\label{line:lis3line4}|
virtualinvoke $r1.<java.lang.Object: void foo()>(); |\label{line:lis3line5}|
r2 = staticinvoke <org.mockito.Mockito: |\label{line:lis3line6}|
				java.lang.Object mock(java.lang.Class)> |\label{line:lis3line7}|
				(class "Ljava/lang/Object;");
virtualinvoke r2.<java.lang.Object: void foo()>(); |\label{line:lis3line9}|
\end{lstlisting}

\section{Basic Declarative Analysis} 
\label{sec:motivating-example-dec}

Again referring to Jimple Listing~\ref{lis:mockMethodIllustrationIR}, this time we ask whether the invocation on Jimple line~\ref{line:lis3line9} satisfies predicate \texttt{isMockInvocation} (facts Listing~\ref{lis:facts}, line~\ref{line:facts-imi}), which we define to hold the analysis result---namely, all mock invocation sites in the program. It does, because of facts lines~\ref{line:facts-vmi}--\ref{line:facts-imv}: Jimple line~\ref{line:lis3line9} contains a virtual method invocation, and the receiver object \texttt{r2} for the invocation on that line satisfies our predicate \texttt{isMockVar}, which holds all mock-containing variables in the program (Section~\ref{sec:dec-doop} provides more details). Predicate \texttt{isMockVar} holds because of lines~\ref{line:facts-arv}--\ref{line:facts-cms}: \texttt{r2} satisfies \texttt{isMockVar} because Jimple line~\ref{line:lis3line6} assigns \texttt{r2} the return value from mock source method \texttt{createMock} (facts line~\ref{line:facts-arv}), and the call to \texttt{createMock} satisfies predicate \texttt{callsMockSource} (facts line~\ref{line:facts-cms}), which requires that the call destination \texttt{createMock} be enumerated as a constant in our 1-ary relation \texttt{MockSourceMethod} (facts line~\ref{line:facts-msm}), and that there be a call graph edge between the method invocation at line~\ref{line:lis3line6} and the mock source method (facts line~\ref{line:facts-cge}).


\begin{lstlisting}[basicstyle=\ttfamily, caption={Facts about invocation \texttt{r2.foo()} in method \texttt{test}.},
basicstyle=\ttfamily, framesep=4.5mm, framexleftmargin=1.0mm, captionpos=b, label=lis:facts, escapechar=!, morekeywords={@Test}]
isMockInvocation(<Object: void foo()>/test/0, 
<Object: void foo()>, test, _. r2). !\label{line:facts-imi}!
|VirtualMethodInvocation(<Object: void foo()>/test/0, !\label{line:facts-vmi}!
|                        <Object: void foo()>, test).
|VirtualMethodInvocation_Base(<Object: void foo()>/test/0, 
|                                  r2).
|isMockVar(r2). !\label{line:facts-imv}!
|-AssignReturnValue(<Mockito: Object mock(Class)>/test/0, !\label{line:facts-arv}!
|                        r2). 
|-callsMockSource(<Mockito: Object mock(Class)>/test/0). !\label{line:facts-cms}!
|MockSourceMethod(<Mockito: Object mock(Class)>). !\label{line:facts-msm}!
|CallGraphEdge(_, <Mockito: Object mock(Class)>/test/0, _, !\label{line:facts-cge}!
|              <Mockito: Object mock(Class)>). 
\end{lstlisting}

\begin{lstlisting}[basicstyle=\ttfamily, caption={This example illustrates a field array container holding mock objects from \textit{setup()} in \texttt{NodeListIteratorTest.java}.},
basicstyle=\ttfamily,language = Java, framesep=4.5mm, framexleftmargin=1.0mm, captionpos=b, label=lis:container, escapechar=|, morekeywords={@Test}]
// Node array to be filled with mock Node instances
private Node[] nodes;
@Test
protected void setUp() throws Exception {
	// create mock Node Instances and 
	// fill Node[] to be used by test cases
	final Node node1 = createMock(Element.class);
	final Node node2 = createMock(Element.class);
	final Node node3 = createMock(Text.class);
	final Node node4 = createMock(Element.class);
	nodes = new Node[] {node1, node2, node3, node4}; |\label{line:storeMocksInArray}|
	// ...
}
\end{lstlisting}

\section{Extensions: arrays and collections} 

While we were designing \textsc{MockDetector}, we observed several cases where developers store mock objects in arrays and collections. Listing~\ref{lis:container} presents method \textit{setUp()} in class \texttt{NodeListIteratorTest} from commons-collections-4.4, where line \ref{line:storeMocksInArray} puts the mock \texttt{Node} objects in the array-typed field \texttt{nodes}. This field is later used in test cases. When the flow function of the dataflow analysis encounters an assignment statement containing an array read or write, it first looks for values (local variables or field reference sources) on the opposite side of the assignment statement---the statement's destination or source, respectively---in the Jimple intermediate representation. It then checks whether any of these local variables or field references have been marked as mock objects in the analysis. If so, the tool marks the local variable or field reference representing the array as an array mock---it propagates the mockness to the array container.

Figure~\ref{fig:arrayMockIllustration} illustrates the process of identifying a mock-containing array, and Listing~\ref{lis:arrayIllustrationIR} displays the Jimple IR of the code in Figure~\ref{fig:arrayMockIllustration}. Our analysis reaches the mock API call on line~\ref{line:lis4line4}--\ref{line:lis4line6}, where it records that \texttt{\$r2} is a mock object---it creates a MockStatus abstraction object with mock bit set to 1 and associates that object with \texttt{\$r2}. The tool then handles the cast expression assigning to \texttt{r1} on line~\ref{line:lis4line7}, giving it the same MockStatus as \texttt{\$r2}. When the analysis reaches line~\ref{line:lis4line9}, it finds an array reference on the left hand side, along with \texttt{r1} stored in the array on the right-hand side of the assignment statement. At that point, it has a MockStatus associated with \texttt{r1}, with the mock bit turned on. It can now deduce that \texttt{\$r3} on the left-hand side is an array container which may hold a mock object. Therefore, \textsc{MockDetector}'s imperative static analysis associates \texttt{\$r3} with a MockStatus with mock-containing array bit (``arrayMock'') set to 1.


\begin{figure}
	\begin{lstlisting}[
	basicstyle=\ttfamily,language = Java, framesep=4.5mm, framexleftmargin=1.0mm, captionpos=b, escapechar=|, morekeywords={@Test}]
	//        mock: |\cmark|     mockAPI: |\cmark|
	Object object1 = createMock(Object.class);
	
	// arrayMock: |\cmark| |$\Leftarrow$| array-write    |~~|  mock: |\cmark|
	objects  |~~|           = new Object[]  |~|  { object1 };
	\end{lstlisting}
	
	\caption{Our static analysis also finds array mocks.}
	\label{fig:arrayMockIllustration}
	
\end{figure}

\begin{lstlisting}[basicstyle=\ttfamily, caption={Jimple Intermediate Representation for the array in Figure~\ref{fig:arrayMockIllustration}.},
basicstyle=\ttfamily, framesep=4.5mm, framexleftmargin=1.0mm, captionpos=b, label=lis:arrayIllustrationIR, escapechar=|, morekeywords={@Test, specialinvoke, virtualinvoke, staticinvoke, newarray}]
java.lang.Object r1, $r2;
java.lang.Object[] $r3;

$r2 = staticinvoke <org.easymock.EasyMock: |\label{line:lis4line4}|
						java.lang.Object createMock(java.lang.Class)>
						(class "java.lang.Object;"); |\label{line:lis4line6}|
r1 = (java.lang.Object) $r2; |\label{line:lis4line7}|
$r3 = newarray (java.lang.Object)[1]; |\label{line:lis4line8}|
$r3[0] = r1;  |\label{line:lis4line9}|
\end{lstlisting}


The declarative analysis uses analogous reasoning, but uses a relation \\ \texttt{isArrayLocalThatContainsMocks} instead of a bit in the MockStatus abstraction.



%======================================================================
\chapter{Technique}
\label{chap:technique}	
%======================================================================

We present two complementary ways of statically computing mock information: an imperative implementation of a dataflow analysis (using the Soot program analysis framework), and a declarative implementation (using the Doop framework). We started this project with the usual imperative approach to implementing a static analysis---in our context, that meant using Soot. Then, when we wanted to experiment with adding more features to the analysis, we decided that this was a good opportunity to learn about Doop's declarative approach as well. We added new features to the Doop implementation and backported them to the Soot implementation. While the core analysis is similar, the different implementation technologies have different affordances. For instance, it is easier for the Doop version to mark a field as mock-containing (we added 3 rules) than for the Soot version to do so. We start by describing each implementation in turn, and conclude this section with the commonalities between the two implementations. Chapter~\ref{chap:evaluation} then presents the results obtained using each technology and compares them. 

% Regarding Gregor's comments on affordances, maybe use advantages or other words?

\section{High Level Definition}
\label{sec:high-level}

Similarly to the dataflow analysis,
the declarative approach propagates mockness from known mock sources, through the statements in the intermediate representation, to potential mock invocation sites.

\section{Imperative Soot Implementation}
\label{sec:soot}
We first describe the Soot-based imperative dataflow analysis to find mock invocation sites. Our tool tracks information from the creation sites through the control-flow graph using a forward dataflow may-analysis---an object is declared a mock if there exists some execution path where it may receive a mock value. Our implementation also understands containers like arrays and collections, and tracks whether containers hold any mock objects. The abstraction marks all contents of collections as potential mocks if it observes any mock object being put into the array or container.


\subsection{Forward Dataflow May Analysis}

%To solve the problem, our tool uses forward may analysis, where it analyzes statements from top to bottom, and to keep variables that are verified to be mocks on any possible path at merged points. \textsc{MockDetector} uses the abstraction 
Our forward dataflow analysis maps values (locals and field references) in the Jimple intermediate representation to our abstraction:
\[ \mathtt{Value} \mapsto \mathtt{MockStatus}. \]
\texttt{MockStatus} records three bits: one for the value being a mock, one for it being an array containing a mock, and one for it being a collection containing a mock. At most one of the three bits may be true for any given value. Not having a mapping in the abstraction is equivalent to mapping to a MockStatus having all three bits false. 

We chose to implement a may-analysis rather than a must-analysis for two reasons: 1) we did not observe any cases where a value was assigned a mock on one branch and a real object on the other branch of an if statement; 2) implementing a must-analysis would not help heuristics to find focal methods, as a must-analysis would rule out fewer mock invocations. Our merge operation is therefore a fairly standard pointwise disjunction of the two incoming values in terms of values and in terms of the 3 bits of \texttt{MockStatus}.

%% \textsc{MockDetector} implements the "may" logic in the following manner: it checks the two in-flows 
%% of \begin{lstlisting}[basicstyle=\ttfamily\small,numbers=none]
%% Map<Value, MockStatus>
%% \end{lstlisting}
%% from two paths. For any variable that is only stored in one map, the key-value pair is directly passed to the out-flow map. For a variable that is a shared key of the two maps, the analysis would update the out-flow's MockStatus by applying the "OR" operation on the "May Mock", "Array Mock", and "Collection Mock" bits from the MockStatus value retrieved from both in-flow maps. 

%% For each statement in a forward flow analysis, we consider two sets: generated set and killed set. In this study, the first set contains the locals that are judged to become mocks, whereas the killed set containing locals that are determined to no longer to be mocks. Equations (1) and (2) illustrates how the inflow and outflow are defined and calculated for each unit: $In(u)$, representing a program point before executing $u$, is the intersection of all outflows after executing each element in immediate predecessor statements of $u$; $Out(u)$, on the other hand, is determined by first removing the killed set from $In(u)$, and union the result with generated set. 

%% \begin{equation}
%% \mathrm{In}(u) = \bigcap_{u' \in preds(u)} \mathrm{Out}(u') 
%% \end{equation}

%% \begin{equation}
%% \mathrm{Out}(u) = (\mathrm{In}(u) - \mathrm{Kill}(u)) \bigcup \mathrm{Gen}(u) 
%% \end{equation}

Our dataflow analysis uses fairly standard gen and kill sets in the flow function. We set bits in \texttt{MockStatus} as follows:

First, the gen set includes pre-analyzed fields containing mock objects defined via annotation (e.g. \texttt{@Mock}), inside a constructor \texttt{<init>}, or in JUnit's \texttt{@Before}/\texttt{setUp()} methods. We discuss the pre-analysis below in Section~\ref{subsec:pre-analysis}. 

Second, it includes local variables assigned from mock-creation source methods, which consist of Mockito's \textit{java.lang.Object mock(java.lang.Class)}, EasyMock's \textit{java.lang.Object createMock(java.lang.Class)}, and PowerMock's \textit{java.lang.Object mock(java.lang.Class)}:
\begin{lstlisting}[basicstyle=\ttfamily\small,numbers=none]
X x = mock(X);
\end{lstlisting}

Third, it includes values assigned from return values of read methods from mock-containing collections or arrays:
\begin{lstlisting}[basicstyle=\ttfamily\small,numbers=none]
// array read;
// r1 is in the in-set as an array mock
X x = r1[0];
// collection read;
// r2 is in the in-set as a collection mock
X x = r2.get(0);
\end{lstlisting}

Fourth, if \texttt{x} is a mock and casted and assigned to \texttt{x\_cast}, then the gen set includes \texttt{x\_cast} (e.g. \texttt{r1} in Listing~\ref{lis:arrayIllustrationIR}):
\begin{lstlisting}[basicstyle=\ttfamily\small,numbers=none]
// x is a mock in the in-set
X x_cast = (X) x;
\end{lstlisting}

Finally, the gen set includes copies of already-flagged mocks:
\begin{lstlisting}[basicstyle=\ttfamily\small,numbers=none]
// x is a mock in the in-set
X y = x;
\end{lstlisting}
The copy-related rules also apply to mock-containing arrays and collections. We add some additional rules for generating mocks that the program reads from collections and arrays, as well as rules for marking arrays and collections as mock-containing. For instance, in the below array write, if the in set has \texttt{r2} as a mock, then the destination \texttt{r1} will be generated as a mock-containing array. Similarly, if \texttt{r3} is a known mock, then the collection \texttt{\$r4} to which it is added (the list of collection add methods is hardcoded) will be generated as a mock-containing collection.
\begin{lstlisting}[basicstyle=\ttfamily\small,numbers=none]
// r2 is in the in set as a mock
r1[0] = r2;
// r3 is in the in set as a mock
$r4.<java.util.ArrayList: boolean add(java.lang.Object)>(r3);
\end{lstlisting}

% nah, we say that above now.
%% In a similar fashion, the gen set will include values that traverse the program's control-flow graph via assignments, from a value that already has mock-containing array or mock-containing collection bit set to true.

%% In this example, $\$r1$ is the immediate receiver from Mockito's mock creation site, whereas $r2$ is the casted expression that gets carried along in the subsequent program. Thus, our tool would include the immediate receivers, and the casted expressions of mock objects into the generation set, in two steps. 

\subsection{Toy Example Revisit - Soot Implementation}
\label{subsec:toy-example-soot}

Let us now revisit the toy example first introduced in Section~\ref{sec:toy-example}. 

Figure~\ref{fig:mockExample} shows how our dataflow analysis works. At the top of the Jimple IR in Listing~\ref{lis:mockExampleIR}, we begin with an empty abstraction (no mapping for any values, equivalent to all bits false for each value) before line~\ref{line:lis3line3}. For the creation of \texttt{\$r1} on line~\ref{line:lis3line3} and~\ref{line:lis3line4}, since the call to the no-arg \texttt{<init>} constructor is not one of our hardcoded mock APIs, our analysis does not declare \texttt{\$r1} to be a mock object. In practice, our abstraction simply does not create an explicit binding for \texttt{\$r1}, instead leaving the mapping empty as it was prior to line~\ref{line:lis3line3}; but it would be equivalent to create a new \texttt{MockStatus} with all bits false and bind it to \texttt{\$r1}. Thus, we may conclude that the invocation \texttt{object1.foo()} on line 5 in Figure~\ref{fig:mockExample} is not known to be a mock invocation. Tying back to our focal methods application, we would not exclude the call to \texttt{foo()} from being a possible focal method.

\begin{figure}
	\begin{lstlisting}[basicstyle=\ttfamily,
	basicstyle=\ttfamily,language = Java, framesep=4.5mm, framexleftmargin=1.0mm, captionpos=b, escapechar=|, morekeywords={@Test}]
	//        mock: |\xmark~\,|    mockAPI: |\xmark|
	Object object1 = new Object();
	
	// mock: |\xmark|
	object1.foo();
	
	//        mock: |\cmark|     mockAPI: |\cmark|
	Object object2 = mock(Object.class);
	
	// mock: |\cmark|
	object2.foo();
	\end{lstlisting}
	%    \includegraphics[width=.25\textwidth]{Images/mockInvocationIllustration.png}
	
	\caption{Our static analysis propagates mockness from sources (e.g. \texttt{mock(Object.class}) to invocations.}
	\label{fig:mockExample}
	
\end{figure}

On the other hand, our imperative analysis sees the mock-creation source methods \\ \texttt{<org.mockito.Mockito: java.lang.Object mock(java.lang.Class)>} on line~\ref{line:lis41line7} and~\ref{line:lis41line8} in the Jimple IR. It thus adds a mapping from local variable \texttt{r2} to a new \texttt{MockStatus} with the mock bit set to true. When the analysis reaches line~\ref{line:lis41line10}, because \texttt{r2} has a mapping in the abstraction with the mock bit being set, \textsc{MockDetector} will deduce that the call on line~\ref{line:lis41line10} is a mock invocation. This implies that the call to method \textit{foo()} on line 11 in Figure~\ref{fig:mockExample} cannot be a focal method.


\begin{lstlisting}[basicstyle=\ttfamily, caption={Jimple Intermediate Representation for the code in Figure~\ref{fig:mockExample}.},
basicstyle=\ttfamily, captionpos=b, label=lis:mockExampleIR, escapechar=|, morekeywords={@Test, specialinvoke, virtualinvoke, staticinvoke}]
java.lang.Object $r1, r2;

$r1 = new java.lang.Object; |\label{line:lis41line3}|
specialinvoke $r1.<java.lang.Object: void <init>()>(); |\label{line:lis41line4}|
virtualinvoke $r1.<java.lang.Object: void foo()>(); |\label{line:lis41line5}|

r2 = staticinvoke <org.mockito.Mockito: |\label{line:lis41line7}|
				java.lang.Object mock(java.lang.Class)> |\label{line:lis41line8}|
				(class "Ljava/lang/Object;");
virtualinvoke r2.<java.lang.Object: void foo()>(); |\label{line:lis41line10}|
\end{lstlisting}

\subsection{Arrays and Containers}

To be explicit about our treatment of arrays and containers: at a read from an array into a local variable where the source array is mock-containing, we declare that the local destination is a mock. At a write of a local variable into an array where the local variable is mock-containing, we declare that the array is mock-containing.

%%  Several test suites use arrays or collection objects to hold mock objects. In this scenario, our tool would consider that mockness propagates out to the container. For instance, say the flow function encounters an array write \begin{lstlisting}[basicstyle=\ttfamily\small,numbers=none]
%%     r1[0] = r2
%% \end{lstlisting} 
%% Then, the tool will look for \texttt{r2} in the abstraction. Once the abstraction pinpoints \texttt{r2} (with mock bit on), it will include \texttt{r1} with mock-containing array bit set to true. (THIS PART STILL FEELS REPETITIVE.)


%% Taking an array as an example, our tool would first look for an array reference in the executing statement, meaning there is a read or a write from an array. If the effect is a STORE to the array, \textsc{MockDetector} would look for variables stored into the array, and check whether any of the variables have been found to be mocks. If so, it would label the array as an array mock, and set the relevant bit in the abstraction to true. Reversely, if is a LOAD effect from the array, \textsc{MockDetector} would check if the array itself has mock-containing array bit on in the abstraction. If so, it will mark the value assigned from the array LOAD with mock bit on, and store it in the abstraction.

%% *** we say more about collections in the related work, we should probably move that to here.

We treat collections analogously. However, while there is one API for arrays---the Java bytecode array load and array store instructions---Java's Collections APIs include, by our count, 60 relevant methods, which we discuss further in Section~\ref{sec:common}. For our purposes here, we use our classification of collection methods to identify collection reads and writes and handle them as we do array reads and writes, except that we say that it is a mock-containing collection, not a mock-containing array.

% The main difference is Java's \texttt{Collection} interface has multiple implementations, which expose different APIs for objects. \textsc{MockDetector} resolves this problem with a manually-constructed pool of read and write method APIs associated with each sub-type of the interface \texttt{java.util.Collection}. It subsequently checks (using the hierarchy) whether collection classes appear in statements containing invoke expression. This is achieved by first determining the declaring class of the invoked method. If the declaring class is of an interface, \textsc{MockDetector} would check whether \texttt{java.util.Collection} is a super-interface for the declaring class. Otherwise, if the declaring class is of a class type, \textsc{MockDetector} would check whether \texttt{java.util.Collection} is a super-interface for any of declaring class's implemented interfaces. If a collection sub-type container is presented, \textsc{MockDetector} would then check if a STORE effect is applied to the container, indicating some object is to be stored in the container. Once the object is determined to be a mock, the collection container variable would immediately be labelled as a collection mock, setting the relevant bit in the abstraction to true. Reversely, if a LOAD effect is applied to the container labelled as a collection mock, the object retrieved from the container will be labelled a mock, and setting the mock bit in the abstraction to true.

\subsubsection{Mock-containing Array Toy Example}

Figure~\ref{fig:arrayMockIllustration} illustrates the process of identifying a mock-containing array, and Listing~\ref{lis:arrayIllustrationIR} displays the Jimple IR of the code in Figure~\ref{fig:arrayMockIllustration}. Our analysis reaches the mock API call on line~\ref{line:lis4line4}--\ref{line:lis4line6}, where it records that \texttt{\$r2} is a mock object---it creates a MockStatus abstraction object with mock bit set to 1 and associates that object with \texttt{\$r2}. The tool then handles the cast expression assigning to \texttt{r1} on line~\ref{line:lis4line7}, giving it the same MockStatus as \texttt{\$r2}. When the analysis reaches line~\ref{line:lis4line9}, it finds an array reference on the left hand side, along with \texttt{r1} stored in the array on the right-hand side of the assignment statement. At that point, it has a MockStatus associated with \texttt{r1}, with the mock bit turned on. It can now deduce that \texttt{\$r3} on the left-hand side is an array container which may hold a mock object. Therefore, \textsc{MockDetector}'s imperative static analysis associates \texttt{\$r3} with a MockStatus with mock-containing array bit (``arrayMock'') set to 1.

\begin{figure}
	\begin{lstlisting}[
	basicstyle=\ttfamily,language = Java, framesep=4.5mm, framexleftmargin=1.0mm, captionpos=b, escapechar=|, morekeywords={@Test}]
	//        mock: |\cmark|     mockAPI: |\cmark|
	Object object1 = createMock(Object.class);
	
	// arrayMock: |\cmark| |$\Leftarrow$| array-write    |~~|  mock: |\cmark|
	objects  |~~|           = new Object[]  |~|  { object1 };
	\end{lstlisting}
	
	\caption{Our static analysis also finds array mocks.}
	\label{fig:arrayMockIllustration}
	
\end{figure}


\begin{lstlisting}[basicstyle=\ttfamily, caption={Jimple Intermediate Representation for the array in Figure~\ref{fig:arrayMockIllustration}.},
basicstyle=\ttfamily, framesep=4.5mm, framexleftmargin=1.0mm, captionpos=b, label=lis:arrayIllustrationIR, escapechar=|, morekeywords={@Test, specialinvoke, virtualinvoke, staticinvoke, newarray}]
java.lang.Object r1, $r2;
java.lang.Object[] $r3;

$r2 = staticinvoke <org.easymock.EasyMock: |\label{line:lis4line4}|
java.lang.Object createMock(java.lang.Class)>
(class "java.lang.Object;"); |\label{line:lis4line6}|
r1 = (java.lang.Object) $r2; |\label{line:lis4line7}|
$r3 = newarray (java.lang.Object)[1]; |\label{line:lis4line8}|
$r3[0] = r1;  |\label{line:lis4line9}|
\end{lstlisting}

\subsection{Pre-Analyses for Field Mocks Defined in Constructors and Before Methods}
\label{subsec:pre-analysis}

A number of our benchmarks define fields as referencing mock objects via EasyMock or Mockito \texttt{@Mock} annotations, or initialize these fields in the \texttt{<init>} constructor or \texttt{@Before} methods (\textit{setUp()} in JUnit 3), which test runners will execute before any test methods from those classes. These mock field or mock-containing container fields are then used in tests. In the Soot implementation, we use two pre-analyses before running the main analysis, under the assumption that fields are rarely mutated in the test cases (and that it is incorrect to do so). We have validated our assumption on benchmarks. An empirical analysis of our benchmarks shows that fewer than 0.3\% of all fields (29/9352) are mutated in tests.
%Table~\ref{tab:mutations} shows a preliminary analysis of field mutation frequency inside test cases---fewer than 0.3\% of fields are mutated in test cases.

The first pre-analysis handles annotated field mocks and field mocks defined in the constructors (\texttt{<init>} methods), while the second pre-analysis handles \texttt{@Before} and \texttt{setUp()} methods. 

\textsc{MockDetector} retrieves all fields in all test classes, and marks fields annotated {\tt @org.mockito.Mock} or {\tt @org.easymock.Mock} as mocks.
% obvious enough that we don't need to belabour this point
% Listing~\ref{lis:annotatedMock} illustrates a Mockito annotated field mock example taken from \texttt{DefaultToolchainManagerTest.java} class in maven-core.

Listing~\ref{lis:fieldMock} depicts an example where instance fields are initialized using field initializers. Java copies such initializers into all class constructors (\texttt{<init>}). To detect such mock-containing fields, we thus simply apply the forward dataflow analysis on all constructors in the test classes prior to running the main analysis, using the same logic that we use to detect mock objects or mock-containing containers in the main analysis. The second pre-analysis handles field mocks defined in \texttt{@Before} methods just like the first pre-analysis handled constructors.

%% Listing~\ref{lis:fieldMock2} illustrates an example where fields are defined as mocks via mock-creation source methods inside the @Before method. Similarly, the field mocks initialized inside the @Before methods are determined by applying the forward dataflow analysis strictly on all @Before methods before the main analysis, where the values  


%% \begin{lstlisting}[basicstyle=\ttfamily, caption={Example for Annotated field mocks from \texttt{DefaultToolchainManagerTest.java} in maven-core.},
%% basicstyle=\scriptsize\ttfamily,language = Java, framesep=4.5mm,
%% framexleftmargin=1mm, captionpos=b, label=lis:annotatedMock]
%% public class DefaultToolchainManagerTest
%% {
%%     @Mock
%%     private Logger logger;
%%     @Mock
%%     private ToolchainFactory toolchainFactory_basicType;
%%     @Mock
%%     private ToolchainFactory toolchainFactory_rareType;

%%     @Before
%%     public void onSetup() throws Exception
%%     {    
%%         // ...
%%         MockitoAnnotations.initMocks( this );
%%         // ...
%%     }
%% }
%% \end{lstlisting}

\begin{lstlisting}[basicstyle=\ttfamily, caption={Example for field mocks defined by field initializations from \texttt{TypeRuleTest.java} in jsonschema2pojo.},
basicstyle=\ttfamily,language = Java, framesep=4.5mm,
framexleftmargin=1mm, captionpos=b, label=lis:fieldMock]
private GenerationConfig config = mock(GenerationConfig.class);
private RuleFactory ruleFactory = mock(RuleFactory.class);
// ...
\end{lstlisting}

\subsection{Interprocedural Support} 

The Heros framework implements IFDS/IDE for program analysis frameworks including Soot. With some effort, it would be possible to rewrite our mock analysis with Heros; however, this would be a more involved process than in the declarative case, where we simply added two rules. In particular, Heros uses a different API in its implementation than Soot. Conceptually, though, it should be no harder to implement an interprocedural Heros analysis than an intraprocedural Soot dataflow analysis.

% not necessary here. you can add it to your thesis.
%% \begin{lstlisting}[basicstyle=\ttfamily, caption={Example for field mocks defined in a \texttt{@Before} method.},
%% basicstyle=\scriptsize\ttfamily,language = Java, framesep=4.5mm,
%% framexleftmargin=1mm, captionpos=b, label=lis:fieldMock2]
%% public class PayRollMockTest {
%%     private EmployeeDB employeeDB;
%%     private BankService bankService;

%%     @Before
%%     public void init() {
%%         // ...
%%         employeeDB = mock(EmployeeDB.class);
%%         bankService = mock(BankService.class);
%%         // ...
%%     }
%% }
%% \end{lstlisting}

\section{Declarative Doop Implementation}
\label{sec:dec-doop}

We next describe the declarative Doop-based technique that \textsc{MockDetector} uses. We implemented this technique by writing Datalog rules. Similarly to the dataflow analysis, the declarative approach propagates mockness from known mock sources, through the statements in the intermediate representation, to potential mock invocation sites.

% Mock Libraries discussed in Common Infrastructure section, perhaps refer to the paragraph in Common Infrastructure section?
The core of the implementation starts by declaring facts for 9 mock source methods manually gleaned from the mock libraries' documentation, as specified through method signatures (e.g. 
\texttt{<org.mockito.Mockito: java.lang.Object mock(java.lang.Class)>}.)
It then declares that a variable {\tt v} satisfies \verb+isMockVar(v)+ if it is assigned from the return value of a mock source, or otherwise traverses the program's interprocedural control-flow graph, through assignments, which may possibly flow through fields, collections, or arrays. Finally, an invocation site is a mock invocation if the receiver object {\tt v} satisfies \verb+isMockVar(v)+.

\begin{lstlisting}[basicstyle=\ttfamily\small,numbers=none,label={lst:core}]
// v = mock()
isMockVar(v) :-
AssignReturnValue(mi, v),
callsMockSource(mi).
// v = (type) from
isMockVar(v) :-
isMockVar(from),
AssignCast(_ /* type */, from, v, _ /* inmethod */).
// v = v1
isMockVar(v) :-
isMockVar(v1),
AssignLocal(v1, v, _).
\end{lstlisting}

The predicates \texttt{AssignReturnValue}, \texttt{AssignCast}, and \texttt{AssignLocal} are provided by Doop, and resemble Java bytecode instructions. Unlike Java bytecode, however, their arguments are explicit. For instance, \texttt{AssignLocal(?from:Var, ?to:Var, ?inmethod:Method)} denotes an assignment statement copying from \texttt{?from} to \texttt{?to} in method \texttt{?inmethod}. (It is Datalog convention to prefix parameters with \texttt{?}s).

We designed the analysis in a modular fashion, such that the interprocedural, collections, arrays, and fields support can all be disabled through the use of \verb+#ifdef+s, which can be specified on the Doop command-line.

\subsection{Toy Example Revisit - Doop Implementation}
\label{subsec:toy-example-doop}

Again referring to Jimple Listing~\ref{lis:mockExampleIR}, this time we ask whether the invocation on Jimple line~\ref{line:lis41line10} satisfies predicate \texttt{isMockInvocation} (facts Listing~\ref{lis:facts}, line~\ref{line:facts-imi}), which we define to hold the analysis result---namely, all mock invocation sites in the program. It does, because of facts lines~\ref{line:facts-vmi}--\ref{line:facts-imv}: Jimple line~\ref{line:lis41line10} contains a virtual method invocation, and the receiver object \texttt{r2} for the invocation on that line satisfies our predicate \texttt{isMockVar}, which holds all mock-containing variables in the program (Section~\ref{sec:dec-doop} provides more details). Predicate \texttt{isMockVar} holds because of lines~\ref{line:facts-arv}--\ref{line:facts-cms}: \texttt{r2} satisfies \texttt{isMockVar} because Jimple line~\ref{line:lis41line7} assigns \texttt{r2} the return value from mock source method \texttt{createMock} (facts line~\ref{line:facts-arv}), and the call to \texttt{createMock} satisfies predicate \texttt{callsMockSource} (facts line~\ref{line:facts-cms}), which requires that the call destination \texttt{createMock} be enumerated as a constant in our 1-ary relation \texttt{MockSourceMethod} (facts line~\ref{line:facts-msm}), and that there be a call graph edge between the method invocation at line~\ref{line:lis41line7} and the mock source method (facts line~\ref{line:facts-cge}).


\begin{lstlisting}[basicstyle=\ttfamily, caption={Facts about invocation \texttt{r2.foo()} in method \texttt{test}.},
basicstyle=\ttfamily, framesep=4.5mm, framexleftmargin=1.0mm, captionpos=b, label=lis:facts, escapechar=!, morekeywords={@Test}]
isMockInvocation(<Object: void foo()>/test/0, 
<Object: void foo()>, test, _. r2). !\label{line:facts-imi}!
|VirtualMethodInvocation(<Object: void foo()>/test/0, !\label{line:facts-vmi}!
|                        <Object: void foo()>, test).
|VirtualMethodInvocation_Base(<Object: void foo()>/test/0, 
|                                  r2).
|isMockVar(r2). !\label{line:facts-imv}!
|-AssignReturnValue(<Mockito: Object mock(Class)>/test/0, !\label{line:facts-arv}!
|                        r2). 
|-callsMockSource(<Mockito: Object mock(Class)>/test/0). !\label{line:facts-cms}!
|MockSourceMethod(<Mockito: Object mock(Class)>). !\label{line:facts-msm}!
|CallGraphEdge(_, <Mockito: Object mock(Class)>/test/0, _, !\label{line:facts-cge}!
|              <Mockito: Object mock(Class)>). 
\end{lstlisting}


\subsection{Interprocedural Support} 

From our perspective, including (context-insensitive) interprocedural support is almost trivial; we only need to add two rules
\begin{lstlisting}[basicstyle=\ttfamily\small,numbers=none]
// v = callee(), where callee's return 
// var is mock
isInterprocMockVar(v) :-
AssignReturnValue(mi, v),
mainAnalysis.CallGraphEdge(_, mi, _, callee),
ReturnVar(v_callee, callee),
isMockVar(v_callee).

// callee(v) results in formal param 
// of callee being mock
isInterprocMockVar(v_callee) :-
isMockVar(v),
ActualParam(n, mi, v),
FormalParam(n, callee, v_callee),
mainAnalysis.CallGraphEdge(_, mi, _, callee),
Method_DeclaringType(callee, callee_class),
ApplicationClass(callee_class).
\end{lstlisting}
using Doop-provided call graph edges (relation \texttt{mainAnalysis.CallGraphEdge}) between the method invocation {\tt mi} and its callee {\tt callee}. The first rule propagates information from callees back to their callers, while the second rule propagates information from callers to callees through parameters. Note that we restrict our analysis to so-called ``application classes'', excluding in particular the Java standard library. We chose to run our conext-insensitive analysis on top of Doop's context-insensitive call graph, but have also reported results with Doop's \texttt{basic-only} analysis, which implements Class Hierarchy Analysis. Mirroring Doop, it would also be possible to add context sensitivity to our analysis, but our results suggest that this would not help much; we'll return to that point in Chapter~\ref{chap:evaluation}.

\subsection{Arrays and Containers} 

Consistent with our analysis being a may-analysis, we define a predicate \\ {\tt isArrayLocalThatContainsMocks} to record local variables pointing to arrays that may contain mock objects. This predicate is true whenever the program under analysis stores a mock variable into an array; we also transfer array-mockness through assignments and casts. When a local variable \texttt{v} is read from a mock-containing array \texttt{c}, then \texttt{v} is marked as a mock variable, as seen in the first rule below. An store of a mock variable \texttt{mv} into an array \texttt{c} causes that array to be marked as \texttt{isArrayLocalThatContainsMocks}. Note that these predicates are mutually recursive. A similar predicate is also applied to containers. We also handle {\tt Collection.toArray} by propagating mockness from the collection to the array.

\begin{lstlisting}[basicstyle=\ttfamily\small,numbers=none]
// v = c[idx]
isMockVar(v) :-
isArrayLocalThatContainsMocks(c),
LoadArrayIndex(c, v, _ /* idx */).

// c[idx] = mv
isArrayLocalThatContainsMocks(c) :-
StoreArrayIndex(mv, c, _ /* idx */),
isMockVar(mv).
\end{lstlisting}

\subsection{Fields} 

Apart from the obvious rule stating that a field which is assigned from a mock satisfies {\tt fieldContainsMock}, we also label fields that have the {\tt org.mockito.Mock} annotation as mock-containing. We declare that a given field \emph{signature} may contain a mock, i.e. the field with a given signature belonging to all objects of a given type. We also support containers stored in fields.

\subsection{Arrays and Fields} 

We also support not just array locals but also array fields. That is, when an array-typed field is assigned from a mock-containing array local, then it is also a mock. And when an array-typed local variable is assigned from a mock-containing array field, then that array local is a mock-containing array.

\section{Common Infrastructure}
\label{sec:common}
We have parameterized our technique with respect to mocking libraries and have instantiated it with respect to the popular Java libraries Mockito, EasyMock, and PowerMock. We also support different versions of JUnit\footnote{\url{https://junit.org}}: 3, and 4+, and we support the Java Collections API. We discuss this parameterization in this subsection.

Both JUnit and mocking libraries rely heavily on reflection to functions, and would normally pose problems for static analyses. In particular, the set of reachable test methods is enumerated at runtime, and the mock libraries create mock objects reflectively. Fortunately, their use of reflection is limited and stylized, and we have designed our analyses to soundly handle these libraries.

\subsection{JUnit and Driver Generation}

JUnit tests are simply methods that developers write in test classes, appropriately annotated (in JUnit 3 by method name starting with ``test'', in 4+ by a \texttt{@Test} annotation). A JUnit test runner uses reflection to find tests. Out of the box, static analysis engines do not see tests as reachable code.

% what about hierarchical drivers?

Thus, to enable static analysis over a benchmark's test suite, our tool uses Soot to generate a driver class for each Java sub-package of the suite (e.g. \\ \texttt{org.apache.ibatis.executor.statement}). In each of these sub-package driver classes, our tool creates a \textit{runall()} method, which invokes all methods within the sub-package that JUnit (either 3 or 4) considers to be public, as well as non-constructor test cases, all surrounded by calls to class-level init/setup and teardown methods. Concrete test methods are particularly easy to call from a driver, as they are specified to have no parameters and are not supposed to rely on any particular ordering. 
Our tool then creates a RootDriver class at the root package level, which invokes the \textit{runall()} method in each sub-package driver class, along with the \texttt{@Test}/\texttt{@Before}/\texttt{@After} methods found in classes located at the root. The drivers that we generate also contain code to catch all checked exceptions declared to be thrown by the unit tests. Both our Soot and Doop implementations use the generated driver classes.

All static frameworks must somehow approximate the set of entry points as appropriate for their targets. For instance, the Wala framework~\cite{wala19:_t} also creates synthetic entry points, but it does this to perform pointer analysis on a program's main code rather than to enumerate the program's test cases.

\subsection{Intraprocedural Analysis} 

The Soot analysis is intraprocedural and the Doop analysis has an intraprocedural version. In both of these cases, we make the unsound (but reasonable for our anticipated use case) assumption that mockness can be dropped between callers and callees: at method entry points, no parameters are assumed to be mocks, and at method returns, the returned object is never a test. Doop's interprocedural version drops this assumption, and instead context-insensitively propagates information from callers to callees and back; we discuss the results of doing so in Chapter~\ref{chap:evaluation}.

Call graphs are useful to our intraprocedural analysis in two ways: first, because they help identify calls to mock source methods (that we identify explicitly); and second, because they come with entry points (which we effectively supply to the call graph using our generated driver, as explained above). We assume that developers do not call inherited versions of mock creation sites. (However, if a call graph is available in Doop, we use it). 

\subsection{Mock Libraries}

Our supported mock libraries take different approaches to instantiating mocks. All of the libraries have methods that generate mock objects; for instance, EasyMock contains the \texttt{createMock()} method. We consider return values from these methods to be mock objects. Additionally, Mockito contains a fluent \texttt{verify()} method which returns a mock object. Finally, Mockito and EasyMock also allow developers to mark fields as \texttt{@Mock}; we treat reads from such fields as mock objects. Both implementations start the analysis with these hard coded facts on mock source methods, as described in the mock libraries' documentation.

\subsection{Containers}

As stated above, we hardcode all relevant methods from the Java Collections API. There are 60 such methods in total, which together account for about 1/6th of the total lines in our Doop analysis. In addition to straightforward get and put methods, we also support iterators, collection copies via constructors, and add-all methods. An iterator can be treated as a copy of the container, with the request of an object from the iterator being tantamount to a container get. An add-all method copies the mock-containing collection bit.


%======================================================================
\chapter{Evaluation}
\label{chap:evaluation}	
%====================================================================== 

%The major difference between the Doop's total runtime and the actual time spent on mock invocation analysis comes from the build of the complete graph. %Add reference for SLOCCount.
In this chapter, we perform a qualitative evaluation of our \textsc{MockDetector} tool in assisting for focal method findings, as well as a quantitative evaluation of the tool in identifying mock objects and finding invocations. We compare the run-time and performance between our Soot and Doop implementations. In addition, we also investigate over four types of base analyses used by Doop, and their effect in mock object findings and efficiencies.

\section{Qualitative Evaluation}

We plan to reproduce Ghafari's algorithm~\cite{ghafari15:_autom} to automatically identify focal methods. However, first, before implementing it, we use manual inspection to evaluate how useful it is to remove mocks from consideration as focal methods. We present two examples in this section to showcase how our tool eliminates method invocations that are definitely not focal methods.

First, let us revisit the example first discussed in Section~\ref{sec:running-example}. Figure~\ref{fig:mockExampleEvaluation} shows the process of locating the mock object \texttt{session} and consequently the mock invocation \texttt{getRequest()} in the example. Both the Soot and Doop implementations report one mock invocation from the test case. Soot outputs the Invoke Expression for \texttt{getRequest()}, whereas Doop outputs the corresponding call-graph edge in the isMockInvocation.csv file. With the assistance of \textsc{MockDetector}, we could remove \texttt{getRequest()} from focal method consideration. We judge \texttt{getToolchainsForType()} to be the focal method because it is the only method invocation remaining after the elimination process. 

Figure~\ref{fig:mockExample2Evaluation} displays the second example. The unit test case is selected from benchmark \textsc{vraptor-core}. It has multiple mock invocations, with the field mock objects defined via the mock annotations. 

Both the Soot and Doop implementations identify all the field mocks, and consequently report three mock invocations from the test case. They also successfully output the Invoke Expressions or the corresponding call-graph edges for the three mock invocations: \texttt{getContentType()}, \texttt{deserializerFor()}, and \texttt{unsupportedMediaType()}. By a process of elimination, we could deduce that \texttt{intercept()} on Line~\ref{line:focal-method} is the focal method. 

Both examples qualitatively demonstrate that our \textsc{MockDetector} tool can help remove invocations on mocks from consideration as focal methods, thus making it easier to identify focal methods. We believe our tool could augment the precision of Ghafari's algorithm. We could also potentially construct a novel elimination-based algorithm. Such an algorithm might be better since it considers more methods that are actually focal methods, like sophisticated getters.

\begin{figure}[h]
	\begin{lstlisting}[
	basicstyle=\ttfamily,language = Java, framesep=4.5mm, framexleftmargin=1.0mm, captionpos=b, escapechar={|}, mathescape=true, morekeywords={@Test}]
	@Test
	public void testMisconfiguredToolchain() throws Exception {
		//        		mock:|\cmark|	 mockAPI:|\cmark|
		MavenSession |\colorbox{olive}{session}| = |\colorbox{teal}{mock}| ( MavenSession.class );
		MavenExecutionRequest req = new DefaultMavenExecutionRequest();
		//     mock invocation:|\cmark| $\Rightarrow$ focal method:|\xmark|
		when( session.|\colorbox{red}{getRequest()}| ).thenReturn( req ); |\label{line:mock}|
		
		ToolchainPrivate[] basics =
		//     					 	focal method:|\cmark|
				toolchainManager.|\colorbox{gray}{getToolchainsForType}|("basic", session); |\label{line:real}|
		
		assertEquals( 0, basics.length );
	}
	\end{lstlisting}
	
	\caption{Qualitative evaluation of removing one mock invocation from focal method consideration.}
	\label{fig:mockExampleEvaluation}
\end{figure}


\begin{figure}[p]
	\begin{lstlisting}[
	basicstyle=\fontsize{10}{12}\ttfamily, language = Java, framesep=4.5mm, framexleftmargin=1.0mm, captionpos=b, escapechar={|}, mathescape=true, morekeywords={@Test}]
	// Fields used in the unit test case
	private DeserializingInterceptor interceptor;
	// mock annotation:|\cmark|
	|\colorbox{teal}{@Mock}| private HttpServletRequest request;
	// mock annotation:|\cmark|
	|\colorbox{teal}{@Mock}| private InterceptorStack stack;
	// mock annotation:|\cmark|
	|\colorbox{teal}{@Mock}| Deserializers deserializers;
	private MethodInfo methodInfo;
	// mock annotation:|\cmark|
	|\colorbox{teal}{@Mock}| Container container;
	// mock annotation:|\cmark|
	|\colorbox{teal}{@Mock}| private Status status;
	
	@Before
	public void setUp() throws Exception {
		MockitoAnnotations.initMocks(this);
		
		methodInfo = new DefaultMethodInfo();
		interceptor = new DeserializingInterceptor(request, deserializers, methodInfo, container, status);
	}
	
	
	@Test
	public void willSetHttpStatusCode415IfThereIsNoDeserializerButIsAccepted() throws Exception {
		//     mock invocation:|\cmark| $\Rightarrow$ focal method:|\xmark|
		when( request.|\colorbox{red}{getContentType()}| ).thenReturn("application/xml");
		//     		mock invocation:|\cmark| $\Rightarrow$ focal method:|\xmark|
		when( deserializers.|\colorbox{red}{deserializerFor}|("application/xml", container) ).thenReturn(null);
		
		//       focal method:|\cmark|
		interceptor.|\colorbox{gray}{intercept}|(stack, consumeXml, null); |\label{line:focal-method}|
		//     		mock invocation:|\cmark| $\Rightarrow$ focal method:|\xmark|
		verify(status).|\colorbox{red}{unsupportedMediaType}|("Unable to handle media type [application/xml]: no deserializer found.");
		verifyZeroInteractions(stack);
	}
	\end{lstlisting}
	
	\caption{Qualitative evaluation of removing multiple mock invocations from focal method consideration.}
	\label{fig:mockExample2Evaluation}
\end{figure}

\section{Description of Benchmark Suite}

We have quantitatively evaluated \textsc{MockDetector} on 8 open-source benchmarks, along with a micro-benchmark that we developed to test our tool. We ran all of our experiments on a 32-core Intel(R) Xeon(R) CPU E5-4620 v2 at 2.60GHz with 128GB of RAM running Ubuntu 16.04.7 LTS.

Table~\ref{tab:runtimes} presents summary information about our benchmarks and run-times, namely the LOC and Soot and Doop analysis run-times for each benchmark. The 9 benchmarks include over 383 kLOC, with 184 kLOC in the test suites, as measured by SLOCCount\footnote{\url{https://dwheeler.com/sloccount/}}. The Soot total time is the amount of time that it takes for Soot to analyze the benchmark and test suite in whole-program mode, including our analyses. The Soot intraprocedural analysis time is the sum of run-times for the main analysis plus two pre-analyses, as described in Section~\ref{sec:soot}. Meanwhile, the reported Doop run-time is from the context-insensitive analysis, while the Doop analysis time for intraprocedural mock invocation analysis is for running the analysis alone based on recorded facts from the benchmark. The total Doop run-time is much slower than the total Soot run-time because Doop always computes a call-graph, which is an expensive operation. We believe that the Doop analysis-only time is also slower because it computes a solution over the entire program, as opposed to Soot, which works one method at a time.

\section{Field Mocks}

We perform an evaluation on the usefulness of our pre-analyses finding field mocks. 
Table~\ref{tab:field-mocks} displays the number of field mock objects that are defined via \texttt{@Mock} annotations, in the constructors, and in the \texttt{@Before}/\texttt{setUp()} methods, respectively. 

We focus on benchmarks such as \textsc{jsonschema2pojo-core}, \textsc{mybatis}, and \textsc{vraptor-core}. These benchmarks have a high number of test-related methods containing mock objects, and have many intraprocedural mock invokes. From the results collected in Table~\ref{tab:field-mocks}, we can tell these benchmarks also prefer to define field mock objects. Instead of repetitively creating the same mock objects in each test case within the same test class, these benchmarks create the field mock objects once and and consequently use them in all the test cases. Such action could help for less code maintenance. In addition, benchmarks like \textsc{bootique} and \textsc{maven-core} also prefer to define field mock objects. Therefore, 5 out of the 8 open-source benchmarks prefer to define field mock objects for the ease of testing. This suggests that our pre-analysis for field mocks described in Section~\ref{subsec:pre-analysis} is indeed a necessary and an effective step for analyzing mock objects and mock invocations in the test suites.

\section{Prevalence of Mocks}

We next investigated the prevalence of mocks. Table~\ref{tab:mocks} presents the number of test-related (Test/Before/After) methods which contain local variables or which access fields that are mocks, mock-containing arrays, or mock-containing collections, as reported by our Soot-based intraprocedural analysis. Across the 8 benchmarks, test-related methods containing local/field mocks or mock-containing containers accounted for 0.35\% to 51.8\% of the total number of test-related methods found in public concrete test classes. Our benchmarks are from different domains and created by different groups of developers. The difference in mock usage reflects their different philosophies and constraints regarding the creation and usage of mock objects in tests. Benchmarks like \textsc{vraptor-core} and \textsc{jsonschema2pojo-core} have more than half of their test-related methods containing mock objects (and mock-containing arrays); in both of these, most field mocks are created via annotations and reused in multiple test cases in the same class.

The core result is in Table~\ref{tab:invokes}, which presents the number of method invocations on mocks detected by our implementations. We present numbers from the imperative intraprocedural Soot implementation, as well as a total of eight versions of the declarative Doop implementation\footnote{bootique, mybatis and vraptor timed out for Doop's 1-object-sensitive analysis without our mock analysis, and we report ``--'' for their times.}: \{ ``basic-only'' (class hierarchy analysis), ``context-insensitive'' (CI), ``context-insensitive-plusplus'' (CIPP), ``1-object-sensitive'' \} Doop base analysis $\times$ \{ intraprocedural, interprocedural \}. 

Note that our declarative and imperative implementations find exactly the same number of intraprocedural mock invocations for 4 of our benchmarks. On the others, the main source of missed mock calls in the Soot implementation is missing support for array- or collection-related functions. Our intraprocedural analysis finds that method invocations on mock objects account for a range from 0.086\% to 16.4\% of the total number of invocations. The two implementations of mock analysis serve to cross validate each other. Often there is only one implementation for a static analysis project, thus it is difficult to judge on the implementation's soundness and correctness. It is possible to formally prove analysis properties, but even then, nothing guarantees conformance of the implementation to the formal description. For this project, I can cross check the results from two implementations, investigate the discrepancies and decide which implementation to further improve on. The improvements on finding intraprocedural mock invocations will be in the future work.

Combining the mock counts result from Table~\ref{tab:mocks} and intraprocedural mock invocations result from Table~\ref{tab:invokes}, we can see that benchmarks such as \textsc{jsonschema2pojo-core}, \textsc{mybatis} and \textsc{vraptor-core} rely quite heavily on mock objects in their tests, which supports our motivation that it is quite necessary to track mock objects and invocations on mocks, to refine the existing algorithm to find focal methods and remove candidates that are definitely not focal methods. 

%--- discuss the numbers for the interprocedural analysis.

\subsection{Intraprocedural vs. Interprocedural}

In Section~\ref{sec:common} we discussed the implementation of our intraprocedural and interprocedural analyses. We can now discuss the effects of these implementation choices on the experimental results. Recall that we chose, unsoundly, to not propagate any information across method calls in the intraprocedural analysis. Thus, the intraproc columns in Table~\ref{tab:invokes} show smaller numbers than the interproc columns, as expected. Note also that there is a sometimes drastic increase from the intraprocedural to the interprocedural result, e.g. from 40 to 1300 for \textsc{flink}. This is because mocks can (especially context-insensitively) propagate from tests to the methods that they call and throughout the main program code. It would be desirable to be able to differentiate test helper methods, which we do want to propagate mocks to, from methods in the main program, which we generally do not want to propagate mocks to. (Although the main program may make mock invocations on objects it is given, we do not want to report these mock invocations from main code.) However, our current analysis infrastructure treats test and main code identically.

\section{Doop Only Analysis}

\subsection{Basic-Only vs. Other Base Analyses}
\label{sec:basic-only-vs-others}

We explored the run-time performance of our 8 declarative analysis variants based on recorded program facts. We used hyperfine\footnote{\url{https://github.com/sharkdp/hyperfine}} to perform 10 benchmarking runs for the Soufflé command of mock analysis, and present the means and standard deviations from the 10 runs in Table~\ref{tab:doop-intra-runtimes} and~\ref{tab:doop-inter-runtimes}.

The running times show that the Doop runs with basic-only base analysis spend more time on mock analysis for most benchmarks than the run-times from the more advanced base analyses. Basic-only is a naive base analysis that generates a much larger call graph than more advanced base analyses, and as it would need to check through more call graph edges, we expect that basic-only would generally spend more time on performing mock analysis. Table~\ref{tab:doop-callgraph-all-counts} highlights the number of source classes and target classes presented in the call graph generated by the four base analyses, which supports the idea that basic-only generates a bigger call graph, and spends most of the extra time checking over unnecessary classes generated in the call graph instead of performing actual mock analysis.

To further investigate the interprocedural running times on benchmarks including \textsc{jsonschema2pojo-core}, \textsc{mybatis}, and \textsc{vraptor-core}, we remove call graph edges that have library classes or dependency classes as source classes for each base analysis, and then count the total number of source classes and target classes in the filtered call graph. As Table~\ref{tab:doop-callgraph-package-counts} shows, if we take the difference of the number of target classes from the number of source classes for each base analysis, the results suggest that the more advanced base analyses (CI, CIPP) reach more target classes from a better defined subset of source classes that are application classes. 

\subsection{Investigate Basic-Only CHA}
\label{subsec:basic-only-cha}

We believe that the basic-only base analyses has some undesired behaviour in the process of generating the CHA. We have discussed with the doop authors. They consider that calls to concrete implementations of abstract methods are not included in CHA. We believe this is not correct from our data collection but it is their assumption. Figure~\ref{fig:edgesToSiblingMethod} shows the call graph edges from the method \textit{isApplicableType} on the class PatternRule to \texttt{java.lang.String fullName()}, which is a sibling method (i.e. a method that has implementations in multiple classes under the same superclass) that is first declared in the abstract base class \texttt{com.sun.codemodel.JType}. From Figure~\ref{fig:edgesToSiblingMethod}, we can tell that the basic-only base analysis only reports edges to the (abstract) method \texttt{com.sun.codemodel.JType.fullName}, whereas the call graph generated by context-insensitive base analysis reports edges to implementation sites of the target method. We believe that this shows that there are missing edges in basic-only from its choice to stop at the abstract level. Combining this information with the total number of interprocedural mock invocations reported for the three benchmarks, we know that CI and CIPP have found notably more mock invocations in the process of searching through the edges to the implementation sites of the sibling methods, and the time spent on this process possibly accounts for the higher mock-analysis running times of CI and CIPP on benchmarks \textsc{jsonschema2pojo-core}, \textsc{mybatis}, and \textsc{vraptor-core}.


\begin{figure}
	\begin{lstlisting}[basicstyle=\ttfamily\scriptsize, numbers=none, framesep=4.5mm, framexleftmargin=1.0mm, captionpos=b, escapechar=|]
	
	Call-graph Edge in Basic-only:
	0	<org.jsonschema2pojo.rules.PatternRule: boolean 
	isApplicableType(com.sun.codemodel.JFieldVar)>/com.sun.codemodel.JClass.fullName/0	
	0	<com.sun.codemodel.JType: java.lang.String fullName()>
	
	Call-graph Edges in CI:
	<<immutable-context>>	<org.jsonschema2pojo.rules.PatternRule: 
	boolean isApplicableType(com.sun.codemodel.JFieldVar)>/com.sun.codemodel.JClass.fullName/0
	<<immutable-context>>	<com.sun.codemodel.JTypeVar: java.lang.String fullName()>
	
	<<immutable-context>>	<org.jsonschema2pojo.rules.PatternRule: 
	boolean isApplicableType(com.sun.codemodel.JFieldVar)>/com.sun.codemodel.JClass.fullName/0	
	<<immutable-context>>	<com.sun.codemodel.JTypeWildcard: java.lang.String fullName()>
	
	<<immutable-context>>	<org.jsonschema2pojo.rules.PatternRule: 
	boolean isApplicableType(com.sun.codemodel.JFieldVar)>/com.sun.codemodel.JClass.fullName/0	
	<<immutable-context>>	<com.sun.codemodel.JNarrowedClass: java.lang.String fullName()>
	
	<<immutable-context>>	<org.jsonschema2pojo.rules.PatternRule: 
	boolean isApplicableType(com.sun.codemodel.JFieldVar)>/com.sun.codemodel.JClass.fullName/0	
	<<immutable-context>>	<com.sun.codemodel.JArrayClass: java.lang.String fullName()>
	
	<<immutable-context>>	<org.jsonschema2pojo.rules.PatternRule: 
	boolean isApplicableType(com.sun.codemodel.JFieldVar)>/com.sun.codemodel.JClass.fullName/0	
	<<immutable-context>>	<com.sun.codemodel.JDefinedClass: java.lang.String fullName()>
	
	<<immutable-context>>	<org.jsonschema2pojo.rules.PatternRule: 
	boolean isApplicableType(com.sun.codemodel.JFieldVar)>/com.sun.codemodel.JClass.fullName/0	
	<<immutable-context>>	<com.sun.codemodel.JDirectClass: java.lang.String fullName()>
	
	<<immutable-context>>	<org.jsonschema2pojo.rules.PatternRule: 
	boolean isApplicableType(com.sun.codemodel.JFieldVar)>/com.sun.codemodel.JClass.fullName/0	
	<<immutable-context>>	<com.sun.codemodel.JCodeModel\$JReferencedClass: java.lang.String fullName()>
	
	\end{lstlisting}
	\caption{The call graph edges to the method \textit{fullName()} from both basic-only and context-insensitive base analyses. The method is declared in abstract class \texttt{com.sun.codemodel.JType} and implemented in its children classes.}
	\label{fig:edgesToSiblingMethod}
	
\end{figure}
% More discussion will be added for the outlier: mybatis and vraptor interprocedural runtimes.

\subsection{Mock Invocations Results by Different Base Analyses}

The four base analyses report the same number of intraprocedural mock invocations in 8 benchmarks. The minor difference in \textsc{vraptor-core} is due to one method (which ought to be present) not showing up in Doop's context-insensitive call-graph. 

From the basic-only's CHA investigation in Section~\ref{subsec:basic-only-cha}, it is probably safe to disregard the interprocedural mock invocation results reported by basic-only.

Analyzing the Doop interprocedural results among the three more advanced base analyses (CI, CIPP and 1-object-sens), they report the same number of interprocedural mock invocations in 5 out of 9 benchmarks, whereas CI and CIPP report the same number of interprocedural mock invocations in 2 more benchmarks where 1-object-sensitive base analysis times out in generating the complete call graph.

% CI and CIPP report more mock invocations in 7 out of the 9 benchmarks compared to basic-only. For benchmarks \textsc{flink-core}, \textsc{jsonschema2pojo-core}, \textsc{mybatis} and \textsc{vraptor-core}, CI and CIPP report significantly more mock invocations than basic-only. Our investigation in Section~\ref{sec:basic-only-vs-others} suggests that such discrepancies are due to the fact that basic-only reports edges to abstract methods and not edges to concrete implementations of these abstract methods. We need to have more discussions with doop authors before giving a definite conclusion, but we believe that the basic-only analysis needs improvement before it can be used to produce a reliable call graph.

\subsection{More Sophisticated Analyses}

In addition, we can observe that CI and CIPP have comparable run-times and mock invocation counts, with CI having slightly higher run-times, both intraprocedurally and interprocedurally. This makes sense, as CIPP is ``context-insensitive with an enhancement for low-hanging fruit: methods that have their params flow to their return value get a methods that have their params flow to their return value get a 1-obj treatment"~\cite{yanniss}. Data presented in Table~\ref{tab:doop-callgraph-all-counts} demonstrates that the call graph sizes are comparable for CI and CIPP, with CI's call graphs generally slightly bigger as expected.

On the other hand, since 1-object-sensitive base analysis is a more sophisticated analysis, it spends more time on building the CHA graph (3 benchmarks ending up with timed-out runs on building the CHA graph). We feed the same input to all four base analyses, but sophisticated analysis like 1-object-sensitive returns a much smaller call-graph. We believe it produces a better defined call-graph and thus executes faster on the actual mock analysis. However, we need to communicate to Doop developers more and have better understandings of the logic involved in these sophisticated analyses before making a conclusion.

%We have successfully executed our Doop analysis with different base analyses. We chose to report numbers for the context-insensitive base analysis here as it matches our own analysis. (It would also be possible, but require significantly more effort, to adapt our analysis to carry around context.)

%--- , indicates the removal of mock invocations from call graph would improve the call graph's accuracy on method coverage for the benchmarks on the high end of the mock invocation percentage. 

% We are currently investigating the performance of intra-procedural basic-only analysis, trying to understand why it would spend more time strictly on mock analysis than CI and CIPP.

%the number of mock invocations correlates with the runtime; taking a bit more effort to compute a better call graph may well pay off in terms of overall analysis time. We suspect that the interprocedural analysis is especially slow for mybatis because we also analyze its 50 dependencies; that count is at the high end among our benchmarks.

\begin{table*}
	\centering
	\caption[Benchmarks' LOC plus Soot and Doop analysis run-times.]{Our suite of 8 open-source benchmarks (8000--117000 LOC) plus our microbenchmark. Soot and Doop analysis run-times.}
	%	\begin{adjustbox}{width=0.1\textwidth}
	\vspace*{.5em}
	\resizebox{\columnwidth}{!}{%
	\begin{tabular}{lrrrrrr}
		\toprule
		Benchmark & Total LOC & Test LOC & \thead{Soot intraproc \\ total time (s)} & \thead{Doop intraproc \\ total time (s)} & \thead{Soot intraproc \\ mock analysis (s)}  & \thead{Doop intraproc \\ mock analysis (s)} \\
		\midrule
		bootique-2.0.B1-bootique           		&  15530   & 8595   & 58  & 2810  &  0.276   & 19.93       \\
		commons-collections4-4.4           		&  65273   & 36318  & 114 & 694   &  0.386   & 14.20       \\
		flink-core-1.13.0-rc1           		&  117310  & 49730  & 341 & 1847  &  0.415   & 27.21        \\
		jsonschema2pojo-core-1.1.1         		&  8233    & 2885   & 313 & 1005   &  0.282   & 29.33       \\
		maven-core-3.8.1   		           		&  38866   & 11104  & 183 & 588   &  0.276   & 19.49        \\
		micro-benchmark         		  		&  954     & 883	& 47  & 387   &  0.130   & 11.73        \\
		mybatis-3.5.6         		  			&  68268   & 46334  & 500 & 4477  &  0.662   & 59.83        \\
		quartz-core-2.3.1        	  			&  35355   & 8423   & 155 & 736   &  0.231   & 21.06     \\
		vraptor-core-3.5.5         	  			&  34244   & 20133  & 371 & 1469  &  0.455   & 34.95      \\
		\bottomrule
		Total         	  						&  384033  & 184405 & 2082 & 14013 &  3.123  & 237.73     \\
	\end{tabular}
	}
	%	\end{adjustbox}
	\label{tab:runtimes}
\end{table*}

\begin{table*}
	\centering
	\caption[Counts of Field Mock Objects.]{Counts of Field Mock Objects defined via \protect \texttt{@Mock} annotation, in the constructors, and in \texttt{@Before} methods, in each benchmark's test suite.}
	%	\begin{adjustbox}{width=0.1\textwidth}
	\vspace*{.5em}
	\resizebox{\columnwidth}{!}{
		\begin{tabular}{lrrrr}
			\toprule
			Benchmark & \thead{\# of Annotated \\ Field Mock Objects} & \thead{\# of Field Mock Objects \\ defined in the \texttt{<init>} constructor}  & \thead{\# of Field Mock Objects \\ defined in @Before methods} \\
			\midrule
			bootique-2.0.B1-bootique           		&  0        &  0    & 8        \\
			commons-collections4-4.4          		&  0        &  0    & 0        \\
			flink-core-1.13.0-rc1           		&  0        &  0    & 0        \\
			jsonschema2pojo-core-1.1.1           	&  26       &  126  & 0        \\
			maven-core-3.8.1	           			&  7        &  0    & 1        \\
			micro-benchmark         		  		&  2        &  0    & 29        \\
			mybatis-3.5.6         		  			&  41       &  0    & 0        \\
			quartz-core-2.3.1         	  			&  0     	&  0    & 0      \\
			vraptor-core-3.5.5         	  			&  263      &  128  & 83       \\
			\bottomrule
		\end{tabular}
	}
	%	\end{adjustbox}
	\label{tab:field-mocks}
\end{table*}

\begin{table*}
	\centering
	\caption[Counts of Mock Objects in Test-Related Methods.]{Counts of Test-Related (Test/Before/After) methods in public concrete test classes, along with counts of mocks, mock-containing arrays, and mock-containing collections, reported by Soot intraprocedural analysis.}
	%	\begin{adjustbox}{width=0.1\textwidth}
	\vspace*{.5em}
	\resizebox{\columnwidth}{!}{
	\begin{tabular}{lrrrr}
		\toprule
		Benchmark & \thead{\# of Test-Related \\ Methods} & \thead{\# of Test-Related \\ Methods with \\ mocks (intra)}  & \thead{\# of Test-Related \\ Methods with \\ mock-containing\\ arrays (intra)} & \thead{\# of Test-Related \\ Methods with \\ mock-containing\\ collections (intra)} \\
		\midrule
		bootique-2.0.B1-bootique           		&  420        &  32  & 7 & 0       \\
		commons-collections4-4.4          		&  1152       &  3   & 1 & 1       \\
		flink-core-1.13.0-rc1           		&  1091       &  4   & 0 & 0       \\
		jsonschema2pojo-core-1.1.1           	&  145        &  76  & 1 & 0       \\
		maven-core-3.8.1	           			&  337        &  24  & 0 & 0       \\
		micro-benchmark         		  		&  59         &  43  & 7 & 25       \\
		mybatis-3.5.6         		  			&  1769       &  330 & 3 & 0       \\
		
		quartz-core-2.3.1         	  			&  218     	  &  7   & 0 & 0      \\
		vraptor-core-3.5.5         	  			&  1119       &  565 & 15 & 0      \\
		\bottomrule
		Total        	  						&  6310       &  1084  & 34 & 26    \\
	\end{tabular}
	}
	%	\end{adjustbox}
	\label{tab:mocks}
\end{table*}

\begin{landscape}
\begin{table*}
	\ra{1.2}
	\centering
	\caption[Number of Mock Invocations in Each Benchmark.]{Number of InstanceInvokeExprs on Mock objects analyzed by Soot and Doop, and Total Number of InstanceInvokeExprs, in each benchmark's test suite. ``--''~=~timed out after 90 minutes. Runs [mybatis, basic-only] and [flink-core, 1-object-sensitive] take close to 90 minutes and sometimes time out.}
	\vspace*{.5em}
	\begin{tabular}{@{}lrrcrrrrcrrrr@{}} \toprule
	Benchmark & \thead{Total \\ Number of \\ Invocations} & \thead{Mock Invokes \\ intraproc (Soot)}
	& \phantom{abc} & \multicolumn{4}{c}{\thead{Mock Invokes \\ intraproc (Doop)}} & \phantom{abc} & \multicolumn{4}{c}{\thead{Mock Invokes \\ interproc (Doop)}}
	\\
	\cmidrule{5-8} \cmidrule{10-13}
	& & & & \thead{basic\\-only} & CI & CIPP & \thead{1-obj\\-sens} & & \thead{basic\\-only} & CI & CIPP & \thead{1-obj\\-sens} \\ \midrule
	\csvreader[head to column names, late after line=\\]
	{Data/DoopAndSootMockCounts.csv}{}%
	{\csvcoli&\csvcolii&\csvcoliii&&\csvcoliv&\csvcolv&\csvcolvi&\csvcolvii& &\csvcolviii&\csvcolix&\csvcolx&\csvcolxi}
	\bottomrule
	\end{tabular}
	\label{tab:invokes}
\end{table*} 
\end{landscape}

%\begin{table*}
%	\centering
%	\caption{Comparison of Number of InstanceInvokeExprs on Mock objects analyzed by Soot and Doop, and Total Number of InstanceInvokeExprs, in each benchmark's test suite. N/A = timed out after 90 minutes.}
%	%	\begin{adjustbox}{width=0.1\textwidth}
%	\begin{tabular}{lrrrrrr}
%		\toprule
%		Benchmark & \thead{Total Number \\ of Invocations} & \thead{Mock Invokes \\ intraproc (Soot)} & \thead{Basic-only, \\ intraproc (Doop)} & \thead{Context-insensitive, \\ intraproc (Doop)} &  \thead{Basic-only, \\ interproc (Doop)} &\thead{Context-insensitive, \\ interproc (Doop)} \\
%		\midrule
%		bootique-2.0.B1-bootique           		&  3366     &  99   & 99    & 99   & 120   & 122    \\
%		commons-collections4-4.4       			&  12753    &  11   & 0     &  3   & 0    & 23   \\
%		flink-core-1.13.0-rc1           		&  11923    &  40   & 40    & 40   & 1262  & 1389   \\
%		jsonschema2pojo-core-1.1.1      	     	&  1896     &  276  & 282   & 282  & 462   & 604   \\
%		maven-core-3.8.1           			&  4072     &  23   & 23    & 23   & 31    & 39  \\
%		microbenchmark         		  		&  471      &  108  & 123   & 123  & 132   & 132   \\
%		mybatis-3.5.6         		  		&  19232    &  575  & N/A   & 577  &  N/A  & 1345     \\
%		quartz-core-2.3.1       	  		&  3436     &  21   & 21    & 21   & 23    & 31    \\
%		vraptor-core-3.5.51        	  		&  5868     &  942  & 963   & 962  & 1301  & 1630   \\
%		\bottomrule
%		Total        	  				&  63017    & 2095  & N/A   & 2130  & N/A  & 5315   \\
%	\end{tabular}
%	%	\end{adjustbox}
%	\label{tab:invokes}
%\end{table*}

%& & \thead{basic\\-only} & $\sigma$ & CI & $\sigma$ & CIPP & $\sigma$ & \thead{1-obj\\-sens} & $\sigma$ \\

\clearpage

\begin{table*}
	\centering
	\caption[Intraprocedural Doop Analysis-Only Run-time.]{Intraprocedural Doop analysis-only run-time (in seconds) after basic-only, context-insensitive, context-insensitive-plusplus and 1-object-sensitive base analyses. \protect\\ ``--''~=~timed out after 90 minutes. Runs [mybatis, basic-only] and [flink-core, 1-object-sensitive] take close to 90 minutes and sometimes time out.}
	\vspace*{.5em}
	\begin{tabular}{@{}lrrrrrrrr} \toprule
		Benchmark & \multicolumn{8}{c}{intraproc}\\
		\cmidrule{2-9}
		& \thead{basic\\-only} & $\sigma$ & CI & $\sigma$ & CIPP & $\sigma$ & \thead{1-obj\\-sens} & $\sigma$ \\ \midrule 
		
		\csvreader[head to column names, late after line=\\]
		{Data/HyperfineRuntimeWithAvgsINTRA.csv}{}%
		{\csvcoli&\csvcolii&{\scriptsize \csvcoliii}&\csvcoliv&{\scriptsize \csvcolv} &\csvcolvi&{\scriptsize \csvcolvii}&\csvcolviii &{\scriptsize \csvcolix}}
		\bottomrule
	\end{tabular}
	\label{tab:doop-intra-runtimes}
\end{table*}

\begin{table*}
	\centering
	\caption[Interprocedural Doop Analysis-Only Run-time.]{Interprocedural Doop analysis-only run-time (in seconds) after basic-only, context-insensitive, context-insensitive-plusplus and 1-object-sensitive base analyses. \protect\\ ``--''~=~timed out after 90 minutes. Runs [mybatis, basic-only] and [flink-core, 1-object-sensitive] take close to 90 minutes and sometimes time out.}
	\vspace*{.5em}
	\begin{tabular}{@{}lrrrrrrrr} \toprule
		Benchmark & \multicolumn{8}{c}{interproc}\\
		\cmidrule{2-9}
		& \thead{basic\\-only} & $\sigma$ & CI & $\sigma$ & CIPP & $\sigma$ & \thead{1-obj\\-sens} & $\sigma$ \\ \midrule 
		
		\csvreader[head to column names, late after line=\\]
		{Data/HyperfineRuntimeWithAvgsINTER.csv}{}%
		{\csvcoli&\csvcolii&{\scriptsize \csvcoliii}&\csvcoliv&{\scriptsize \csvcolv} &\csvcolvi&{\scriptsize \csvcolvii}&\csvcolviii &{\scriptsize \csvcolix}}
		\bottomrule
	\end{tabular}
	\label{tab:doop-inter-runtimes}
\end{table*}


\begin{table*}
	\centering
	\caption[Call Graph Statistics.]{Call graph statistics: total number of source classes and target classes from interprocedural Doop analysis with basic-only, context-insensitive, context-insensitive-plusplus, and 1-object-sensitive base analyses. ``--''~=~timed out after 90 minutes. Runs [mybatis, basic-only] and [flink-core, 1-object-sensitive] take close to 90 minutes and sometimes time out.}
	\vspace*{.5em}
	\begin{tabular}{@{}lrrrrcrrrr} \toprule
		Benchmark & \multicolumn{4}{c}{Source Classes} & \phantom{abc} & \multicolumn{4}{c}{Target Classes}
		\\
		\cmidrule{2-5} \cmidrule{7-10}
		& \thead{basic\\-only} & CI & CIPP & \thead{1-obj\\-sens} & & \thead{basic\\-only} & CI & CIPP & \thead{1-obj\\-sens} \\ \midrule
		\csvreader[head to column names, late after line=\\]
		{Data/CallGraphEdgeClassCounts.csv}{}%
		{\csvcoli&\csvcolii&\csvcoliii&\csvcoliv&\csvcolv&&\csvcolvi&\csvcolvii&\csvcolviii&\csvcolix}
		\bottomrule
	\end{tabular}
	\label{tab:doop-callgraph-all-counts}
\end{table*}

\begin{table*}
	\centering
	\caption[Call Graph Statistics Excluding Dependencies.]{Call graph statistics: total number of source classes that are application classes (i.e., excluding classes from dependencies or libraries) and total number of target classes reached from application classes by interprocedural Doop analysis with basic-only, context-insensitive, context-insensitive-plusplus, and 1-object-sensitive base analyses. ``--''~=~timed out after 90 minutes. Runs [mybatis, basic-only] and [flink-core, 1-object-sensitive] take close to 90 minutes and sometimes time out.}
	\vspace*{.5em}
	\begin{tabular}{@{}lrrrrcrrrr} \toprule
		Benchmark & \multicolumn{4}{c}{Source Classes} & \phantom{abc} & \multicolumn{4}{c}{Target Classes}
		\\
		\cmidrule{2-5} \cmidrule{7-10}
		& \thead{basic\\-only} & CI & CIPP & \thead{1-obj\\-sens} & & \thead{basic\\-only} & CI & CIPP & \thead{1-obj\\-sens} \\ \midrule
		\csvreader[head to column names, late after line=\\]
		{Data/CallGraphEdgePackageClassCounts.csv}{}%
		{\csvcoli&\csvcolii&\csvcoliii&\csvcoliv&\csvcolv&&\csvcolvi&\csvcolvii&\csvcolviii&\csvcolix}
		\bottomrule
	\end{tabular}
	\label{tab:doop-callgraph-package-counts}
\end{table*}

%\begin{table*}
%	\centering
%	\caption{Doop analysis-only runtime after basic-only, context-insensitive, context-insensitive-plusplus and 1-object-sensitive base analyses. N/A = timed out after 90 minutes.}
%	%	\begin{adjustbox}{width=0.1\textwidth}
%	\begin{tabular}{lrrrrrr}
%		\toprule
%		Benchmark & \thead{Basic-only, \\ intraproc (s)} & \thead{Context-insensitive, \\ intraproc (s)} & \thead{Basic-only, \\ interproc (s)}  & \thead{Context-insensitive, \\ interproc (s)}  \\
%		\midrule
%		bootique-2.0.B1-bootique           		& 15.71  & 16.81 &  24.26    &  20.20     \\
%		commons-collections4-4.4           		& 17.42  & 12.26 &  21.79    &  15.36        \\
%		flink-core-1.13.0-rc1           		& 24.67  & 25.30 &  71.67    &  66.10         \\
%		jsonschema2pojo-core-1.1.1         		& 25.98  & 26.27 &  42.14    &  39.21         \\
%		maven-core-3.8.1   		        	& 18.01  & 16.34 &  25.49    &  22.09          \\
%		micro-benchmark         			& 10.97  & 10.50 &  12.51    &  12.53        \\
%		mybatis-3.5.6         		  		&  N/A   & 51.25 &   N/A     & 183.86          \\
%		quartz-core-2.3.1        	  		& 17.72  & 19.83 &  22.99    &  21.14        \\
%		vraptor-core-3.5.5         	  		& 22.10  & 23.81 &  66.73    & 146.09       \\
%		\bottomrule
%	\end{tabular}
%	%	\end{adjustbox}
%	\label{tab:doop-runtimes}
%\end{table*}


%======================================================================
\chapter{Related Work}
\label{chap:related}
%====================================================================== 

We discuss related work in the areas of declarative versus imperative static analysis, treatment of containers, and taint analysis.

\section{Imperative vs Declarative}

Kildall contributed perhaps the first dataflow analysis~\cite{kildall73:_unified_approac_global_progr_optim} as the concept is understood today, describing an algorithm for intraprocedural constant propagation and common subexpression elimination. His algorithm, operating on the program graph, is described in quite imperative pseudocode (and proven to terminate). In some sense, implementing algorithms imperatively is the default, and doesn't need further discussion, except to point out that program analysis frameworks such as Soot~\cite{Vallee-Rai:1999:SJB:781995.782008} provide libraries that can ease the implementation burden.

To our knowledge, Corsini et al did some of the first work in declarative program analysis~\cite{corsini93:_effic}; however, that work performed abstract interpretation on (tiny) logic programs rather than imperative programs. Dawson et al~\cite{dawson96:_pract_progr_analy_using_gener} did similar work. Around the same time, Reps proposed~\cite{Reps1995} a declarative analysis to perform on-demand versions of interprocedural program analyses, which is similar to what we have here; however, we compute all of the analysis results rather than performing an on-demand analysis. CodeQuest by Hajijev et al~\cite{hajiyev06} also allows developers to perform AST-level code queries using a declarative query language. {\sc Dimple$^+$}~\cite{benton07:_inter_scalab_declar_progr_analy}\cite[Chapter 3]{benton08:_fast_effec_progr_analy_objec_level_paral} by Benton and Fischer may be closest to what we are advocating as the declarative analysis approach. While Benton's dissertation presents a simple {\sc Dimple$^+$} implementation of Andersen's points-to analysis, the {\sc Dimple$^+$} work does not have Doop's sophisticated pointer analysis available to it. Soufflé, by Scholz et al~\cite{scholz16:_fast_large_scale_progr_analy_datal}, advocates for declarative static analysis (but without comparing it directly to an imperative approach as we do here), and presents performance optimizations needed to achieve this goal.
Finally, Doop~\cite{bravenboer09:_stric_declar_specif_sophis_point_analy}, which is now primarily implemented with a Soufflé backend, is perhaps the most powerful extant declarative program analysis, and focuses on expressing sophisticated pointer analyses in Datalog. 



% implementation note: Reps's approach is much more complicated than what we have in Doop. Perhaps Doop's use of SSA and simulation of phi nodes allows it to use much simpler rules, or maybe it's the specific analyses that are being implemented. e.g. for Doop, which computes an overapproximation, merging the two branches using the virtual phi node (simulated as "x = phi(x1,x2) => x = x1; x = x2") works just fine.

In terms of comparing implementations, Prakash et al~\cite{prakash21:_effec_progr_repres_point_analy} compare pointer analyses as provided by Doop and Wala; in some sense, the present work is similar to that work in that both works compare two frameworks. However, that work compares empirical results from two families of pointer analysis implementations (and finds that the specific intermediate representation used doesn't change the results much), while we discuss the process of implementing a static analysis declaratively versus imperatively. Like us, they note that Doop is difficult to incorporate into a program transformation framework (it works better in standalone mode) while Wala's results are readily available; a similar result applies to any result that a Soot-based data flow analysis produces as compared to a Doop-based declarative analysis.

\section{Treatment of Containers}

In this work, we use coarse-grained abstractions for containers, consistent with the approach from Chu et al~\cite{chu12:_collec_disjoin_analy}. In our experience, test cases do not perform sophisticated container manipulations where it would be necessary to track exactly which elements of a container are mocks. Were such an analysis necessary, we could use the fine-grained container client analysis by Dillig et at~\cite{dillig11:_precis_reason_progr_using_contain}.

\section{Taint Analysis}

Like many other static analyses, our mock analysis can be seen as a variant of a static taint analysis: sources are mock creation methods, while sinks are method invocations. There are no sanitizers in our case. However, for a taint analysis, there is usually a small set of sink methods, while in our case, every method invocation in a test method is a potential sink. Additionally, the goal of our analysis (detecting possible mocks) is different in that it is not security-sensitive, so the balance between false positives and false negatives is different---it is less critical to not miss any potential mock invocations, whereas missing a whole class of tainted methods would often be unacceptable.


%======================================================================
\chapter{Discussion}
\label{chap:discussion}
%====================================================================== 

Having described our imperative and declarative approaches to implementing mock analysis, we now comment on the strengths and weaknesses of these two approaches. We hope that our discussion will help future designers of source code analyses and frameworks.

\section{Subsequent Use of Results} 

Doop is a standalone tool. It depends on other tools to provide input, but provides output in the form of \texttt{csv} files, whose content can be matched to the program source, if a subsequent analysis has the appropriate internal representation. On the other hand, Soot is a compiler framework. Thus, using the Soot analysis results in a subsequent compilation phase is quite easy. Doop works quite well for producing analysis results, and not quite as well for using these results in a compilation process. Our Soot analysis also doesn't need to process the whole program for itself to produce the analysis results that we're interested in here---our intraprocedural analysis can use the existing in-memory representation and pass it on to the next phase, while Doop reads the whole program, throws it away, and leaves nothing for the next compilation phase. 

\section{Expressiveness vs Concision}

In~\cite{bravenboer09:_stric_declar_specif_sophis_point_analy}, Bravenboer and Smaragdakis point out that:
\begin{quote}
	Even conceptually clean program analysis algorithms that
	rely on mutually recursive definitions often get transformed
	into complex imperative code for implementation purposes.
\end{quote}
The presentation of the declarative approach in Chapter~\ref{chap:technique} could meaningfully include direct excerpts from the Datalog; including Java code is rarely meaningful, as there is too much boilerplate in that language.

The declarative approach takes 237 non-comment lines, compared to about 533 non-comment lines for the main part of the imperative approach, which is a significant point in favour of Doop. A head-to-head comparison is tricky, as the imperative approach also uses pre-analyses which are not present in the declarative approach.

We comment on the reasons for using helper analyses in the imperative version and not the declarative version. Recall that the helper analyses pre-computed information about 1) mock annotations and 2) constructors and setup methods. The mock annotations are an inessential difference; they could be computed on the fly in the imperative version, as they are in the declarative version. As for the constructors: when thinking imperatively, it is more intuitive to explicitly order the computations for constructors before regular test methods. On the other hand, thinking declaratively, it is more natural to use mutual recursion to declare a dependency on the results of previous computations for fields (our relation \texttt{isCollectionFieldThatContainsMocks} in particular) than to declare an explicit ordering. There is a small semantic difference in the two implementations, as the declarative implementation does not require field writes to be confined to constructors and setup methods; in this particular case, we empirically verified that the imperative assumption was almost always satisfied.

We also contrast how we store the abstraction in the two versions. The imperative version uses a standard dataflow analysis abstraction (three bits per local variable/field reference), along with an explicitly specified merge operator, while the declarative version uses one relation for each of the three bits. Propagating and merging data happens automatically in Doop.
% There is something going on with doop and kills, but I don't know enough of what's happening to meaningfully comment on it. Intersection-based analyses seem to be possible, because there is e.g. IntraproceduralMustPointTo. There's also something to do with phi nodes and redefinitions, but I can't clearly express it.

Another difference between the declarative and imperative versions is in the support for interprocedural analysis. As stated earlier, in Chapter~\ref{chap:technique}, the declarative version implements a context-insensitive interprocedural analysis while the imperative version is intraprocedural. The choice of intraprocedural versus interprocedural depends strongly on the particular analysis being implemented. Implementing the interprocedural analysis declaratively was impressively easy, while it is significantly more challenging to implement an interprocedural analysis in Soot, requiring the use of Heros~\cite{soap12ifds}, an additional framework. On the other hand, the Heros implementation would be IFDS-based and be context-sensitive; it would be somewhat harder to upgrade our context-insensitive implementation to a context-sensitive Doop implementation.

It is easier to add instrumentation, e.g. timers, to the imperative version than the declarative version. Doop contains some built-in timers, but it is unclear how to add new ones.

\section{Development Velocity}

To help the reader calibrate our descriptions, we describe our experience levels with Soot and Doop. I initially had little Soot experience, developed the Soot implementation through countless hours of testing and debugging, and took guidance and suggestions from my supervisor, who is an early code contributor to the Soot framework. My supervisor, the co-author of the submitted conference research paper, developed the Doop implementation.

Soot is a mature program analysis framework and many of the common sticking points have, over the years, been addressed by the developers. Nevertheless, it can be intimidating to start working with Soot. Our experience with Doop is that it is overall robust, yet still being actively developed (i.e. occasionally, at the start, some daily snapshots didn't work with some versions of the underlying Soufflé engine). There is more documentation for Soot than for Doop, although even for Doop, it is often possible to scrape together answers to one's questions from the source code and the online documentation. Finding the right API (or relation, in Doop) to use can be challenging for both Soot and Doop; it's impossible for us to fairly compare them, due to our different experiences with Soot and Doop.

We thank the Doop developers for their timely and helpful answers to our questions; developer or community support is necessary to successfully use Doop (or, for that matter, most research-grade program analysis frameworks, including Soot).

Most of the time, adding a feature to the declarative version (e.g. field support) required an evening of work. This typically happened first; the declarative version is better for cleanly describing some approximation of the desired behaviour. Somewhat to our surprise, it was then possible to fast-follow with the imperative version, which ended up not taking much more than an evening to implement either. We believe that the existence of the declarative specification helped with designing the imperative version.

The declarative version was still subject to the combinatorial feature interaction problem; for instance, when we added support for fields and containers, we also specifically needed to add support for containers stored in fields.

Debugging is an inevitable part of any development process, including this one; declarative languages are no proof against debugging. Some Doop errors were just frustrating, e.g. hardcoding a syntactically incorrect method signature for a collection method. Other times, better type system enforcement in Datalog, and in particular, identifying relations that are unsatisfiable due to type conflicts, would help. Soot errors are typical programming errors.

Iteration speed can help with more effective debugging. On some benchmarks, Soot iterations could finish in under a minute, while Doop analysis-only iterations could finish in 10 seconds (but we didn't know that at the time). To expand on that: while developing our analysis, we ran our analysis together with the main analysis, and recomputed the main analysis every time we iterated. Yet, Doop supports running add-on analyses like ours, in isolation, after the main analysis terminates. If we were doing it again, we would develop our analysis as a run-after analysis. Running with the main analysis requires at least a 2.5-minute iteration time due to the necessity of re-compiling and re-running the entire analysis every time the analysis changes, while running an analysis after the main analysis can take 10 seconds, as mentioned above. Setting up the analysis to run after the main analysis is trickier and requires understanding of Doop which we did not have until late in the process. 

As stated above, instrumentation is easier in Soot than in Doop, and that extends to printing debug information and using traditional debugging tools, which works as well for Soot as traditional debugging does in general. To debug the Doop analysis, we resorted to outputting relevant relations after a Doop run and manually pinpointing which facts were missing or extraneous. Because Doop uses Soot to generate program facts, understanding Soot in particular and compilers in general was invaluable while developing the Doop implementation---we also looked at the Soot intermediate representation to understand what analysis information was flowing to which intermediate variables.


\section{Future Work} 
\label{sec:future-work}

In this project, we presented two implementations for detecting mock objects and tracking mock invocations in test suites. There are multiple areas require further investigation to improve \textsc{MockDetector}'s quality:

\begin{itemize}
	\item We will work on Soot interprocedural implementations in our tool for a more complete mock analysis. It will be useful to compare and further understand the two approaches analyzing mock objects.
	\item The current interprocedural Doop implementation could not differentiate method invocations in the test suites from the ones in the source code. We will work on modifying Doop's implementation to only report results from the test suites for move accurate results.
	\item We plan to build construct an automated focal method analysis tool. It aims to help quantitatively evaluate our \textsc{MockDetector} tool in assisting for the focal method findings.
\end{itemize}


\section{Summary} 

We would conclude that, especially with the knowledge we have now gained about Doop, prototyping in Doop is easier than in Soot, but that it is no panacea; it remains subject to the feature interaction problem as well as debugging. Additionally, trying to add certain functional behaviours to the Doop implementation, such as timers, can be challenging.

Overall, in this thesis, we have described a case study of the \textsc{MockDetector} static analysis, which we intend to use for further static analyses of test cases. We have implemented \textsc{MockDetector} twice---once imperatively in Soot, and once declaratively in Doop---and characterized its performance on a test suite. Finally, we have discussed our experience implementing this analysis twice, and pointed out the benefits and disadvantages of the imperative and declarative approaches for writing static analyses.

%----------------------------------------------------------------------
% END MATERIAL
% Bibliography, Appendices, Index, etc.
%----------------------------------------------------------------------

% Bibliography

% The following statement selects the style to use for references.  
% It controls the sort order of the entries in the bibliography and also the formatting for the in-text labels.
\bibliographystyle{plain}
% This specifies the location of the file containing the bibliographic information.  
% It assumes you're using BibTeX to manage your references (if not, why not?).
\cleardoublepage % This is needed if the "book" document class is used, to place the anchor in the correct page, because the bibliography will start on its own page.
% Use \clearpage instead if the document class uses the "oneside" argument
\phantomsection  % With hyperref package, enables hyperlinking from the table of contents to bibliography             
% The following statement causes the title "References" to be used for the bibliography section:
\renewcommand*{\bibname}{References}

% Add the References to the Table of Contents
\addcontentsline{toc}{chapter}{\textbf{References}}

\bibliography{uw-ethesis}
% Tip: You can create multiple .bib files to organize your references. 
% Just list them all in the \bibliogaphy command, separated by commas (no spaces).

% The following statement causes the specified references to be added to the bibliography even if they were not cited in the text. 
% The asterisk is a wildcard that causes all entries in the bibliographic database to be included (optional).
\nocite{*}
%----------------------------------------------------------------------

% Appendices

% The \appendix statement indicates the beginning of the appendices.
\appendix
% Add an un-numbered title page before the appendices and a line in the Table of Contents
\chapter*{APPENDICES}
\addcontentsline{toc}{chapter}{APPENDICES}
% Appendices are just more chapters, with different labeling (letters instead of numbers).
\chapter{Field Mutation Analysis}
\label{AppendixA}

\begin{table*}[b]
	\centering
	\caption{Counts of unit test cases containing MustMock object, counts of unit test cases with array containing mock, and counts of unit test cases with collection containing mock in the 3 benchmarks.}
	%	\begin{adjustbox}{width=0.1\textwidth}
	\begin{tabular}{lr}
		\toprule
		Benchmark &  \thead{ Total \# of Fields Mutated \\ in Test Cases / Total \#  of Fields} \\
		\midrule
		bootique-2.0.B1-bootique           			&  0 / 271       \\
		commons-collections4-4.4           			&  3 / 697       \\
		flink-core-1.13.0-rc1           			&  8 / 2675       \\
		jsonschema2pojo-core-1.1.1           		&  0 / 228       \\
		maven-core-3.8.1           					&  0 / 765       \\
		microbenchmark           					&  5 / 32       \\
		mybatis-3.5.6           					&  0 / 2618       \\
		quartz-core-2.3.1           				&  2 / 878       \\
		vraptor-core-3.5.5           				&  10 / 1193       \\
		\bottomrule
		Total           				&  29 / 9352       \\
	\end{tabular}
	%	\end{adjustbox}
	\label{tab:mutations}
\end{table*}

% GLOSSARIES (Lists of definitions, abbreviations, symbols, etc. provided by the glossaries-extra package)
% -----------------------------
%\printglossaries
\cleardoublepage
\phantomsection		% allows hyperref to link to the correct page

%----------------------------------------------------------------------
\end{document} % end of logical document
