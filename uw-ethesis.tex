%======================================================================
% University of Waterloo Thesis Template for LaTeX 
% Last Updated November, 2020 
% by Stephen Carr, IST Client Services, 
% University of Waterloo, 200 University Ave. W., Waterloo, Ontario, Canada
% FOR ASSISTANCE, please send mail to request@uwaterloo.ca

% DISCLAIMER
% To the best of our knowledge, this template satisfies the current uWaterloo thesis requirements.
% However, it is your responsibility to assure that you have met all requirements of the University and your particular department.

% Many thanks for the feedback from many graduates who assisted the development of this template.
% Also note that there are explanatory comments and tips throughout this template.
%======================================================================
% Some important notes on using this template and making it your own...

% The University of Waterloo has required electronic thesis submission since October 2006. 
% See the uWaterloo thesis regulations at
% https://uwaterloo.ca/graduate-studies/thesis.
% This thesis template is geared towards generating a PDF version optimized for viewing on an electronic display, including hyperlinks within the PDF.

% DON'T FORGET TO ADD YOUR OWN NAME AND TITLE in the "hyperref" package configuration below. 
% THIS INFORMATION GETS EMBEDDED IN THE PDF FINAL PDF DOCUMENT.
% You can view the information if you view properties of the PDF document.

% Many faculties/departments also require one or more printed copies. 
% This template attempts to satisfy both types of output. 
% See additional notes below.
% It is based on the standard "book" document class which provides all necessary sectioning structures and allows multi-part theses.

% If you are using this template in Overleaf (cloud-based collaboration service), then it is automatically processed and previewed for you as you edit.

% For people who prefer to install their own LaTeX distributions on their own computers, and process the source files manually, the following notes provide the sequence of tasks:
 
% E.g. to process a thesis called "mythesis.tex" based on this template, run:

% pdflatex mythesis	-- first pass of the pdflatex processor
% bibtex mythesis	-- generates bibliography from .bib data file(s)
% makeindex         -- should be run only if an index is used 
% pdflatex mythesis	-- fixes numbering in cross-references, bibliographic references, glossaries, index, etc.
% pdflatex mythesis	-- it takes a couple of passes to completely process all cross-references

% If you use the recommended LaTeX editor, Texmaker, you would open the mythesis.tex file, then click the PDFLaTeX button. Then run BibTeX (under the Tools menu).
% Then click the PDFLaTeX button two more times. 
% If you have an index as well,you'll need to run MakeIndex from the Tools menu as well, before running pdflatex
% the last two times.

% N.B. The "pdftex" program allows graphics in the following formats to be included with the "\includegraphics" command: PNG, PDF, JPEG, TIFF
% Tip: Generate your figures and photos in the size you want them to appear in your thesis, rather than scaling them with \includegraphics options.
% Tip: Any drawings you do should be in scalable vector graphic formats: SVG, PNG, WMF, EPS and then converted to PNG or PDF, so they are scalable in the final PDF as well.
% Tip: Photographs should be cropped and compressed so as not to be too large.

% To create a PDF output that is optimized for double-sided printing: 
% 1) comment-out the \documentclass statement in the preamble below, and un-comment the second \documentclass line.
% 2) change the value assigned below to the boolean variable "PrintVersion" from " false" to "true".

%======================================================================
%   D O C U M E N T   P R E A M B L E
% Specify the document class, default style attributes, and page dimensions, etc.
% For hyperlinked PDF, suitable for viewing on a computer, use this:
\documentclass[letterpaper,12pt,titlepage,oneside,final]{book}
 
% For PDF, suitable for double-sided printing, change the PrintVersion variable below to "true" and use this \documentclass line instead of the one above:
%\documentclass[letterpaper,12pt,titlepage,openright,twoside,final]{book}

% Some LaTeX commands I define for my own nomenclature.
% If you have to, it's easier to make changes to nomenclature once here than in a million places throughout your thesis!
\newcommand{\package}[1]{\textbf{#1}} % package names in bold text
\newcommand{\cmmd}[1]{\textbackslash\texttt{#1}} % command name in tt font 
\newcommand{\href}[1]{#1} % does nothing, but defines the command so the print-optimized version will ignore \href tags (redefined by hyperref pkg).
%\newcommand{\texorpdfstring}[2]{#1} % does nothing, but defines the command
% Anything defined here may be redefined by packages added below...

% This package allows if-then-else control structures.
\usepackage{ifthen}
\newboolean{PrintVersion}
\setboolean{PrintVersion}{false}
% CHANGE THIS VALUE TO "true" as necessary, to improve printed results for hard copies by overriding some options of the hyperref package, called below.

%\usepackage{nomencl} % For a nomenclature (optional; available from ctan.org)
\usepackage{amsmath,amssymb,amstext} % Lots of math symbols and environments
\usepackage[pdftex]{graphicx} % For including graphics N.B. pdftex graphics driver 

% Hyperlinks make it very easy to navigate an electronic document.
% In addition, this is where you should specify the thesis title and author as they appear in the properties of the PDF document.
% Use the "hyperref" package 
% N.B. HYPERREF MUST BE THE LAST PACKAGE LOADED; ADD ADDITIONAL PKGS ABOVE
\usepackage[pdftex,pagebackref=false]{hyperref} % with basic options
%\usepackage[pdftex,pagebackref=true]{hyperref}
		% N.B. pagebackref=true provides links back from the References to the body text. This can cause trouble for printing.
\hypersetup{
    plainpages=false,       % needed if Roman numbers in frontpages
    unicode=false,          % non-Latin characters in Acrobat’s bookmarks
    pdftoolbar=true,        % show Acrobat’s toolbar?
    pdfmenubar=true,        % show Acrobat’s menu?
    pdffitwindow=false,     % window fit to page when opened
    pdfstartview={FitH},    % fits the width of the page to the window
%    pdftitle={uWaterloo\ LaTeX\ Thesis\ Template},    % title: CHANGE THIS TEXT!
%    pdfauthor={Author},    % author: CHANGE THIS TEXT! and uncomment this line
%    pdfsubject={Subject},  % subject: CHANGE THIS TEXT! and uncomment this line
%    pdfkeywords={keyword1} {key2} {key3}, % list of keywords, and uncomment this line if desired
    pdfnewwindow=true,      % links in new window
    colorlinks=true,        % false: boxed links; true: colored links
    linkcolor=blue,         % color of internal links
    citecolor=green,        % color of links to bibliography
    filecolor=magenta,      % color of file links
    urlcolor=cyan           % color of external links
}
\ifthenelse{\boolean{PrintVersion}}{   % for improved print quality, change some hyperref options
\hypersetup{	% override some previously defined hyperref options
%    colorlinks,%
    citecolor=black,%
    filecolor=black,%
    linkcolor=black,%
    urlcolor=black}
}{} % end of ifthenelse (no else)

\usepackage[automake,toc,abbreviations]{glossaries-extra} % Exception to the rule of hyperref being the last add-on package
% If glossaries-extra is not in your LaTeX distribution, get it from CTAN (http://ctan.org/pkg/glossaries-extra), 
% although it's supposed to be in both the TeX Live and MikTeX distributions. There are also documentation and 
% installation instructions there.

% Setting up the page margins...
% uWaterloo thesis requirements specify a minimum of 1 inch (72pt) margin at the
% top, bottom, and outside page edges and a 1.125 in. (81pt) gutter margin (on binding side). 
% While this is not an issue for electronic viewing, a PDF may be printed, and so we have the same page layout for both printed and electronic versions, we leave the gutter margin in.
% Set margins to minimum permitted by uWaterloo thesis regulations:
\setlength{\marginparwidth}{0pt} % width of margin notes
% N.B. If margin notes are used, you must adjust \textwidth, \marginparwidth
% and \marginparsep so that the space left between the margin notes and page
% edge is less than 15 mm (0.6 in.)
\setlength{\marginparsep}{0pt} % width of space between body text and margin notes
\setlength{\evensidemargin}{0.125in} % Adds 1/8 in. to binding side of all 
% even-numbered pages when the "twoside" printing option is selected
\setlength{\oddsidemargin}{0.125in} % Adds 1/8 in. to the left of all pages when "oneside" printing is selected, and to the left of all odd-numbered pages when "twoside" printing is selected
\setlength{\textwidth}{6.375in} % assuming US letter paper (8.5 in. x 11 in.) and side margins as above
\raggedbottom

% The following statement specifies the amount of space between paragraphs. Other reasonable specifications are \bigskipamount and \smallskipamount.
\setlength{\parskip}{\medskipamount}

% The following statement controls the line spacing.  
% The default spacing corresponds to good typographic conventions and only slight changes (e.g., perhaps "1.2"), if any, should be made.
\renewcommand{\baselinestretch}{1} % this is the default line space setting

% By default, each chapter will start on a recto (right-hand side) page.
% We also force each section of the front pages to start on a recto page by inserting \cleardoublepage commands.
% In many cases, this will require that the verso (left-hand) page be blank, and while it should be counted, a page number should not be printed.
% The following statements ensure a page number is not printed on an otherwise blank verso page.
\let\origdoublepage\cleardoublepage
\newcommand{\clearemptydoublepage}{%
  \clearpage{\pagestyle{empty}\origdoublepage}}
\let\cleardoublepage\clearemptydoublepage

% Define Glossary terms (This is properly done here, in the preamble and could also be \input{} from a separate file...)
% Main glossary entries -- definitions of relevant terminology
\newglossaryentry{computer}
{
name=computer,
description={A programmable machine that receives input data,
               stores and manipulates the data, and provides
               formatted output}
}

% Nomenclature glossary entries -- New definitions, or unusual terminology
\newglossary*{nomenclature}{Nomenclature}
\newglossaryentry{dingledorf}
{
type=nomenclature,
name=dingledorf,
description={A person of supposed average intelligence who makes incredibly brainless misjudgments}
}

% List of Abbreviations (abbreviations type is built in to the glossaries-extra package)
\newabbreviation{aaaaz}{AAAAZ}{American Association of Amateur Astronomers and Zoologists}

% List of Symbols
\newglossary*{symbols}{List of Symbols}
\newglossaryentry{rvec}
{
name={$\mathbf{v}$},
sort={label},
type=symbols,
description={Random vector: a location in n-dimensional Cartesian space, where each dimensional component is determined by a random process}
}
\makeglossaries

%======================================================================
%   L O G I C A L    D O C U M E N T
% The logical document contains the main content of your thesis.
% Being a large document, it is a good idea to divide your thesis into several files, each one containing one chapter or other significant chunk of content, so you can easily shuffle things around later if desired.
%======================================================================
\begin{document}

%----------------------------------------------------------------------
% FRONT MATERIAL
% title page,declaration, borrowers' page, abstract, acknowledgements,
% dedication, table of contents, list of tables, list of figures, nomenclature, etc.
%----------------------------------------------------------------------
% T I T L E   P A G E
% -------------------
% Last updated October 23, 2020, by Stephen Carr, IST-Client Services
% The title page is counted as page `i' but we need to suppress the
% page number. Also, we don't want any headers or footers.
\pagestyle{empty}
\pagenumbering{roman}

% The contents of the title page are specified in the "titlepage"
% environment.
\begin{titlepage}
        \begin{center}
        \vspace*{1.0cm}

        \Huge
        {\bf MockDetector: Detecting and tracking mock objects in unit tests }

        \vspace*{1.0cm}

        \normalsize
        by \\

        \vspace*{1.0cm}

        \Large
        Qian Liang \\

        \vspace*{3.0cm}

        \normalsize
        A thesis \\
        presented to the University of Waterloo \\ 
        in fulfillment of the \\
        thesis requirement for the degree of \\
        Master of Applied Science \\
        in \\
        Electrical and Computer Engineering \\

        \vspace*{2.0cm}

        Waterloo, Ontario, Canada, 2021 \\

        \vspace*{1.0cm}

        \copyright\ Qian Liang 2021 \\
        \end{center}
\end{titlepage}

% The rest of the front pages should contain no headers and be numbered using Roman numerals starting with `ii'
\pagestyle{plain}
\setcounter{page}{2}

\cleardoublepage % Ends the current page and causes all figures and tables that have so far appeared in the input to be printed.
% In a two-sided printing style, it also makes the next page a right-hand (odd-numbered) page, producing a blank page if necessary.

 
% E X A M I N I N G   C O M M I T T E E (Required for Ph.D. theses only)
% Remove or comment out the lines below to remove this page
\begin{center}\textbf{Examining Committee Membership}\end{center}
  \noindent
The following served on the Examining Committee for this thesis. The decision of the Examining Committee is by majority vote.
  \bigskip
  
  \noindent
\begin{tabbing}
Internal-External Member: \=  \kill % using longest text to define tab length
Supervisor(s): \> Patrick Lam \\
\> Associate Professor \\
\> University of Waterloo \\
\end{tabbing}
  \bigskip
  
  \noindent
  \begin{tabbing}
Internal-External Member: \=  \kill % using longest text to define tab length
Internal Member: \> Derek Rayside \\
\> Associate Professor \\
\> University of Waterloo \\
\end{tabbing}
  \bigskip
  
  \noindent
\begin{tabbing}
Internal-External Member: \=  \kill % using longest text to define tab length
Internal Member: \> Gregor Richards \\
\> Lecturer \\
\> University of Waterloo \\
\end{tabbing}
  \bigskip

\cleardoublepage

% D E C L A R A T I O N   P A G E
% -------------------------------
 \begin{center}\textbf{Author's Declaration}\end{center}
  
 \noindent
This thesis consists of material all of which I authored or co-authored: see Statement of Contributions included in the thesis. This is a true copy of the thesis, including any required final revisions, as accepted by my examiners.
  \bigskip
  
  \noindent
I understand that my thesis may be made electronically available to the public.

\cleardoublepage

% S T A T E M E N T   O F   C O N T R I B U T I O N S
% ---------------------------------------------------
\begin{center}\textbf{Statement of Contributions}\end{center}

This thesis consists of all chapters written for conference research paper submission, with minor word changes and styling updates.

Qian Liang was the sole author for Chapters~\ref{chap:motivating-example},~\ref{chap:evaluation}, and Sections~\ref{sec:soot},~\ref{sec:common}, which were written under the supervision of Dr. Patrick Lam. I was responsible for developing the imperative Soot implementation for mock analysis, carrying out data collection and analysis from both imperative Soot and declarative Doop's implementations.

Qian Liang and Dr. Patrick Lam were the co-authors for Chapters~\ref{chap:introduction},~\ref{chap:related} and~\ref{chap:discussion}. 

Dr. Patrick Lam was the sole author for Section~\ref{sec:dec-doop}.

\cleardoublepage

% A B S T R A C T
% ---------------

\begin{center}\textbf{Abstract}\end{center}

% Used the abstract from research paper submission for now. Will modify the text.
Unit testing is a widely used tool in modern software development processes. A well-known issue in writing tests is handling dependencies: creating usable objects for dependencies is often complicated. Developers must therefore often introduce mock objects to stand in for dependencies during testing. 

Test suites are an increasingly-important component of the source code of a software system. We believe that the static analysis of test suites, alongside the systems under test, can enable developers to better characterize the behaviours of existing test suites, thus guiding further test suite analysis and manipulation. However, because mock objects are created using reflection, they confound existing static analysis techniques. At present, it is impossible to statically distinguish methods invoked on mock objects from methods invoked on real objects. Static analysis tools therefore currently cannot determine which dependencies' methods are actually tested, versus mock methods being called.

In this thesis, we introduce MockDetector, a technique to identify mock objects and track method invocations on mock objects. We first built a Soot-based imperative dataflow analysis implementation of MockDetector. Then, to quickly prototype new analysis features and to explore declarative program analysis, we created a Doop-based declarative analysis, added features to it, and ported them back to the Soot-based analysis. Both analyses handle common Java mock libraries' APIs for creating mock objects and propagate this information through test cases. Following our observations of tests in the wild, we have added special-case support for arrays and collections holding mock objects. On our suite of 8 open-source benchmarks, our imperative dataflow analysis approach reported 2,095 invocations on mock objects, whereas our declarative dataflow approach reported 2,130 invocations on mock objects, out of a total number of 63,017 method invocations in test suites; across benchmarks, mock invocations accounted for a range from 0.086\% to 16.4\% of the total invocations. Removing confounding mock invocations from consideration as focal methods can improve the precision of focal method analysis, a key prerequisite to further analysis of test cases. %Both implementations have reported the same number of intra-procedural mock invocations on 4 out of the 8 open source benchmarks analyzed. 

% results part to be updated

\cleardoublepage

% A C K N O W L E D G E M E N T S
% -------------------------------

\begin{center}\textbf{Acknowledgements}\end{center}

I would like to thank to my advisor, Professor Patrick Lam, without whom the thesis
would not have been possible. I would also like to thank to the readers of the thesis,
Professor Gregor Richards and Professor Derek Rayside.
\cleardoublepage

% D E D I C A T I O N
% -------------------

\begin{center}\textbf{Dedication}\end{center}

I dedicate the thesis to my family, who have always supported me during my study and life at University of Waterloo.
\cleardoublepage

% T A B L E   O F   C O N T E N T S
% ---------------------------------
\renewcommand\contentsname{Table of Contents}
\tableofcontents
\cleardoublepage
\phantomsection    % allows hyperref to link to the correct page

% L I S T   O F   F I G U R E S
% -----------------------------
\addcontentsline{toc}{chapter}{List of Figures}
\listoffigures
\cleardoublepage
\phantomsection		% allows hyperref to link to the correct page

% L I S T   O F   T A B L E S
% ---------------------------
\addcontentsline{toc}{chapter}{List of Tables}
\listoftables
\cleardoublepage
\phantomsection		% allows hyperref to link to the correct page

% Change page numbering back to Arabic numerals
\pagenumbering{arabic}

 

%----------------------------------------------------------------------
% MAIN BODY
% We suggest using a separate file for each chapter of your thesis.
% Start each chapter file with the \chapter command.
% Only use \documentclass or \begin{document} and \end{document} commands in this master document.
% Tip: Putting each sentence on a new line is a way to simplify later editing.
%----------------------------------------------------------------------
%======================================================================
\chapter{Introduction}
\label{chap:introduction}
%======================================================================
\doublespacing

Mock objects~\cite{beck02:_test_driven_devel} are a common idiom in
unit tests for object-oriented systems.  They allow developers to test objects that 
rely on other objects, likely from different components, or that are simply complicated 
to build for testing purposes (e.g. a database).

While mock objects are an invaluable tool for developers, their use
complicates the static analysis and manipulation of test case source code, one of our planned future
research directions. Such static analyses can help IDEs provide better
support to test case writers; enable better static estimation of test coverage
(avoiding mocks); and detect focal methods in test cases.

Ghafari et al discussed the notion of a focal method~\cite{ghafari15:_autom} for a test case---the method
whose behaviour is being tested---and presented a heuristic for determining focal methods.
By definition, the focal method's receiver object cannot be a mock object.
Ruling out mock invocations can thus improve the accuracy of focal method detection and
enable better understanding of a test case's behaviour.

Mock objects are difficult to analyze statically because, at the bytecode level,
a call to a mock object statically resembles a call to the real object (as
intended by the designers of mock libraries).
A naive static analysis attempting to be sound would have to include all of 
the possible behaviours of the actual object (rather than the mock) when analyzing such code. 
Such potential but unrealizable behaviours obscure the true behaviour 
of the test case.

We have designed a static analysis, \textsc{MockDetector}, which identifies
mock objects in test cases. It starts from a list of mock object creation sites
(our analyses include hardcoded APIs for common mocking libraries EasyMock, Mockito, and PowerMock). 
It then propagates mockness
through the test and identifies invocation sites as (possibly) mock.
Given this analysis result, a subsequent analysis
can ask whether a given variable in a test case contains a mock or not, and
whether a given invocation site is a call to a mock object or not. We have
evaluated \textsc{MockDetector} on a suite of 8 benchmarks plus a microbenchmark. 
We have cross-checked results across the two implementations and manually inspected
the results on our microbenchmark, to ensure that the results are as expected.

Taking a broader view, we believe that helper static analyses like \textsc{MockDetector} 
can aid
in the development of more useful static analyses. These analyses can
encode useful domain properties; for instance, in our case, properties
of test cases. By taking a domain-specific approach, analyses can extract
useful facts about programs that would otherwise be difficult to establish.

We make the following contributions in this thesis:
\begin{itemize}
	\item We designed and implemented two variants of a static mock detection algorithm, one as a dataflow analysis implemented imperatively (using Soot) and the other declaratively (using Doop).
	\item We evaluate both the relative ease-of-implementation and precision of the imperative and declarative approaches, both intraprocedurally and interprocedurally (for Doop). % potentially intraprocedural as well
	\item We characterize our benchmark suite (8 open-source benchmarks, 184 kLOC) with respect to their use of mock objects, finding that 1084 out of 6310 unit tests use intraprocedurally-detectable mock objects, and that there are a total of 2095 method invocations on mock objects. %We further identify how powerful an analysis is required to identify mock object use---adding fields and collections adds X mock objects, while interprocedural techniques add Y mock objects.
\end{itemize}
At a higher level, we see the thesis as making both a contribution and a meta-contribution to
problems in source code analysis. The contribution, mock detection, enables more accurate analyses
of test cases, which account for a significant fraction of modern codebases. The meta-contribution,
comparing analysis approaches, will help future researchers decide how to best solve their
source code analysis problems. In brief, the declarative approach allows users to quickly prototype, stating their properties
concisely, while the imperative approach is more amenable to use in program transformation; we return
to this question in Chapter~\ref{chap:discussion}.

%======================================================================
\chapter{Motivation}
\label{chap:motivation}
%====================================================================== 

In this chapter, we illustrate how \textsc{MockDetector} finds variables containing mock objects and mock invocations in unit tests---invocations with mock objects as receiver objects. Our tool identifies invocations on variables which have been assigned an object flowing from a mock creation site either using a forward dataflow may-analysis (Soot-based analysis) or by solving specified declarative constraints (Doop-based analysis). Before diving into the current project, let us take a step back and talk about what led us to the research on detecting mock objects and tracking mock invocations.

\section{Preliminary Research}

\begin{figure}[h]
	\centering
	\tikzstyle{block} = [rectangle, draw, 
text width=3em, text centered, rounded corners, minimum height=2em]

\begin{tikzpicture}[>=stealth',shorten >=1pt,auto,node distance=2cm,
semithick,initial text=]

\node[normal,block,text width=7em]   (A)              {\textit{Assembled-Chronology}};

\node[normal, xshift=-8em, yshift=-3em, text width=6em, text centered,rounded corners] (B) [below left of=A] {\nodepart{one} \textit{Buddhist-Chronology} 
\nodepart{two} \textit{withZone()}:\\TESTED $\checkmark$ };

\node[normal, xshift=0.5em, yshift=-3em, text width=6em, text centered,rounded corners] (D) [below left of=A] {\nodepart{one} \textit{GJ-Chronology} 
	\nodepart{two} \textit{withZone()}:\\TESTED $\checkmark$ };

\node[normal,xshift=1em,text width=1em,draw=none]           (G) [below of=A] {\ldots};

\node[normal, xshift=10em, yshift=-3em, text width=6.3em, text centered,rounded corners] (S) [below left of=A] {\nodepart{one} \textit{Strict-Chronology} 
	\nodepart{two} \textit{withZone()}:\\UNTESTED~$\times$};


\draw (B) -> (A);
\draw (G) -> (A);
\draw (D) -> (A);
\draw (S) -> (A); 

\end{tikzpicture} 
	\caption[Caption for SiblingClass untested method.]{joda-time contains superclass \textit{AssembledChronology} and subclasses \textit{BuddhistChronology}, \textit{GJChronology}, \textit{StrictChronology}, and others. Method \textit{withZone()} is tested in \textit{Buddhist-} and \textit{GJChronology} but not \textit{StrictChronology}.\footnotemark}
	\label{fig:hierarchyView}
\end{figure}

\footnotetext{The content of this figure have been incorporated within a NIER paper published by IEEE in 2020 IEEE International Conference on Software Maintenance and Evolution (ICSME), available online: https://ieeexplore.ieee.org/xpl/conhome/9240597/proceeding[doi: 10.1109/ICSME46990.2020.00075]. Qian Liang and Patrick Lam, "SiblingClassTestDetector: Finding Untested Sibling Functions"}

Prior to the development for \textsc{MockDetector}, we were working on a project to detect untested functions that have analogous implementations in sibling classes (they share a common superclass), where at least one of the related implementations are tested. The overall goal of that project is to reduce untested code. Though testing could not guarantee desired program behaviour, developers certainly know nothing about any untested code. Since the sibling methods share the same specification, it is likely that a unit test case covering for one sibling class's implementation may also work for the untested after small modifications, potentially increasing the statement coverage and consequently having a better chance to gain behavioural insight of the benchmark.

Figure~\ref{fig:hierarchyView} illustrates such an example from an open-source benchmark, joda-time (version 2.10.5). The abstract class \textit{AssembledChronology} inherits the specification of method \textit{withZone()} from its parent class, which is not shown in the Figure. \textit{AssembledChronology}'s subclasses \textit{BuddhistChronology}, \textit{GJChronology}, \textit{StrictChronology}, and many others are at the same hierarchy level, which are defined as sibling classes. These sibling classes all have an implementation of \textit{withZone()}; however, the \textit{withZone()} implementations in \textit{BuddhistChronology} and \textit{GJChronology} are tested, whereas the implementation in \textit{StrictChronology} is not.

As the research progressed, we encountered the test case in Listing~\ref{lis:siblingMethodCall}, and realized that it would be a necessary step to remove method invocations on mock objects from the call graph generated by existing static analysis frameworks, as otherwise we may mistakenly treat such test case as the one covering for the tested sibling method, since the existing static analyses tools could not distinguish method invocations on mock objects from method invocations on real objects. 

\begin{lstlisting}[basicstyle=\ttfamily, caption={This code snippet illustrates an example from commons-collections4, where the method \textit{addAll()} invoked on the mock object \texttt{c} could be mistreated as a focal method being covered by existing static analysis frameworks.},
basicstyle=\ttfamily,language = Java, framesep=4.5mm, escapechar=|,
framexleftmargin=1.0mm, captionpos=b, label=lis:siblingMethodCall, morekeywords={@Test}]

@Test
public void addAllForIterable() {
final Collection<Integer> inputCollection = createMock(Collection.class);
...
final Collection<Number> c = createMock(Collection.class);
...
expect(c.addAll(inputCollection)).andReturn(false);
}
\end{lstlisting}

\section{Running Example: Detecting Mock Objects and Mock Invocations}

To motivate our work, consider Listing~\ref{lis:mockCall}, which presents a unit test case from the Maven project. Line~\ref{line:mock} calls \textit{getRequest()}, invoking it on the mock object \texttt{session}. Line~\ref{line:real} then calls \textit{getToolchainsForType()}, which happens to be the focal method whose behaviour is being tested in this test case. At the bytecode level, the two method invocations are indistinguishable with respect to mockness; to our knowledge, current static analysis tools cannot easily tell the difference between the method invocation on a mock object on line~\ref{line:mock} and the method invocation on a real object on line~\ref{line:real}. Given mockness information, an IDE could provide better suggestions. The uncertainty about mockness would confound a naive static analysis that attempts to identify focal methods. For instance, Ghafari et al's heuristic~\cite{ghafari15:_autom} would fail on this test, as it returns the last mutator method in the object under test, and the focal method here is an accessor.

\begin{lstlisting}[basicstyle=\ttfamily, caption={This code snippet illustrates an example from maven-core, where calls to both the focal method \texttt{getToolchainsForType()} and to mock \texttt{session}'s \texttt{getRequest()} method occur in the test \textit{testMisconfiguredToolchain()}.},
basicstyle=\ttfamily,language = Java, framesep=4.5mm, escapechar=|,
framexleftmargin=1.0mm, captionpos=b, label=lis:mockCall, morekeywords={@Test}]
@Test
public void testMisconfiguredToolchain() throws Exception {
	MavenSession session = mock( MavenSession.class );
	MavenExecutionRequest req = new DefaultMavenExecutionRequest();
	when( session.getRequest() ).thenReturn( req ); |\label{line:mock}|

	ToolchainPrivate[] basics =
			toolchainManager.getToolchainsForType("basic", session); |\label{line:real}|

	assertEquals( 0, basics.length );
}
\end{lstlisting}

\section{Basic Dataflow Analysis} 

The dataflow-analysis-based \textsc{MockDetector} implementation maintains an abstraction mapping values (local variables or field references) in Soot's Jimple intermediate representation (IR)~\cite{Vallee-Rai:1999:SJB:781995.782008} to \texttt{MockStatus}, which holds three bits monitoring each value's status being a mock, a mock-containing array, and a mock-containing collection, respectively. The declarative version, which we do not explain in this section, maintains relations that track the same bits.

Figure~\ref{fig:mockMethodIllustration} shows how our dataflow analysis works, and Listing~\ref{lis:mockMethodIllustrationIR} below shows the Jimple IR of the code in Figure~\ref{fig:mockMethodIllustration}. At the top of the Jimple IR, we begin with an empty abstraction (no mapping for any values, equivalent to all bits false for each value) before line~\ref{line:lis3line3}. For the creation of \texttt{\$r1} on line~\ref{line:lis3line3} and~\ref{line:lis3line4}, since the call to the no-arg \texttt{<init>} constructor is not one of our hardcoded mock APIs, our analysis does not declare \texttt{\$r1} to be a mock object. In practice, our abstraction simply does not create an explicit binding for \texttt{\$r1}, instead leaving the mapping empty as it was prior to line~\ref{line:lis3line3}; but it would be equivalent to create a new \texttt{MockStatus} with all bits false and bind it to \texttt{\$r1}. Thus, we may conclude that the invocation \texttt{object1.foo()} on line 5 in Figure~\ref{fig:mockMethodIllustration} is not known to be a mock invocation. Tying back to our focal methods application, we would not exclude the call to \texttt{foo()} from being a possible focal method.

\begin{figure}
	\begin{lstlisting}[basicstyle=\ttfamily,
	basicstyle=\ttfamily,language = Java, framesep=4.5mm, framexleftmargin=1.0mm, captionpos=b, escapechar=|, morekeywords={@Test}]
	//        mock: |\xmark~\,|    mockAPI: |\xmark|
	Object object1 = new Object();
	
	// mock: |\xmark|
	object1.foo();
	
	//        mock: |\cmark|     mockAPI: |\cmark|
	Object object2 = mock(Object.class);
	
	// mock: |\cmark|
	object2.foo();
	\end{lstlisting}
	%    \includegraphics[width=.25\textwidth]{Images/mockInvocationIllustration.png}
	
	\caption{Our static analysis propagates mockness from sources (e.g. \texttt{mock(Object.class}) to invocations.}
	\label{fig:mockMethodIllustration}
	
\end{figure}

On the other hand, our imperative analysis sees the mock creation API \\ \texttt{<org.mockito.Mockito: java.lang.Object mock(java.lang.Class)>} on line~\ref{line:lis3line6} and~\ref{line:lis3line7} in the Jimple IR. It thus adds a mapping from local variable \texttt{r2} to a new \texttt{MockStatus} with the mock bit set to true. When the analysis reaches line~\ref{line:lis3line9}, because \texttt{r2} has a mapping in the abstraction with the mock bit being set, \textsc{MockDetector} will deduce that the call on line~\ref{line:lis3line7} is a mock invocation. This implies that the call to method \textit{foo()} on line 11 in Figure~\ref{fig:mockMethodIllustration} cannot be a focal method.


\begin{lstlisting}[basicstyle=\ttfamily, caption={Jimple Intermediate Representation for the code in Figure~\ref{fig:mockMethodIllustration}.},
basicstyle=\ttfamily, captionpos=b, label=lis:mockMethodIllustrationIR, escapechar=|, morekeywords={@Test, specialinvoke, virtualinvoke, staticinvoke}]
java.lang.Object $r1, r2;

$r1 = new java.lang.Object; |\label{line:lis3line3}|
specialinvoke $r1.<java.lang.Object: void <init>()>(); |\label{line:lis3line4}|
virtualinvoke $r1.<java.lang.Object: void foo()>(); |\label{line:lis3line5}|
r2 = staticinvoke <org.mockito.Mockito: |\label{line:lis3line6}|
				java.lang.Object mock(java.lang.Class)> |\label{line:lis3line7}|
				(class "Ljava/lang/Object;");
virtualinvoke r2.<java.lang.Object: void foo()>(); |\label{line:lis3line9}|
\end{lstlisting}

\section{Basic Declarative Analysis} 
\label{sec:motivating-example-dec}

Again referring to Jimple Listing~\ref{lis:mockMethodIllustrationIR}, this time we ask whether the invocation on Jimple line~\ref{line:lis3line9} satisfies predicate \texttt{isMockInvocation} (facts Listing~\ref{lis:facts}, line~\ref{line:facts-imi}), which we define to hold the analysis result---namely, all mock invocation sites in the program. It does, because of facts lines~\ref{line:facts-vmi}--\ref{line:facts-imv}: Jimple line~\ref{line:lis3line9} contains a virtual method invocation, and the receiver object \texttt{r2} for the invocation on that line satisfies our predicate \texttt{isMockVar}, which holds all mock-containing variables in the program (Section~\ref{sec:dec-doop} provides more details). Predicate \texttt{isMockVar} holds because of lines~\ref{line:facts-arv}--\ref{line:facts-cms}: \texttt{r2} satisfies \texttt{isMockVar} because Jimple line~\ref{line:lis3line6} assigns \texttt{r2} the return value from mock source method \texttt{createMock} (facts line~\ref{line:facts-arv}), and the call to \texttt{createMock} satisfies predicate \texttt{callsMockSource} (facts line~\ref{line:facts-cms}), which requires that the call destination \texttt{createMock} be enumerated as a constant in our 1-ary relation \texttt{MockSourceMethod} (facts line~\ref{line:facts-msm}), and that there be a call graph edge between the method invocation at line~\ref{line:lis3line6} and the mock source method (facts line~\ref{line:facts-cge}).


\begin{lstlisting}[basicstyle=\ttfamily, caption={Facts about invocation \texttt{r2.foo()} in method \texttt{test}.},
basicstyle=\ttfamily, framesep=4.5mm, framexleftmargin=1.0mm, captionpos=b, label=lis:facts, escapechar=!, morekeywords={@Test}]
isMockInvocation(<Object: void foo()>/test/0, 
<Object: void foo()>, test, _. r2). !\label{line:facts-imi}!
|VirtualMethodInvocation(<Object: void foo()>/test/0, !\label{line:facts-vmi}!
|                        <Object: void foo()>, test).
|VirtualMethodInvocation_Base(<Object: void foo()>/test/0, 
|                                  r2).
|isMockVar(r2). !\label{line:facts-imv}!
|-AssignReturnValue(<Mockito: Object mock(Class)>/test/0, !\label{line:facts-arv}!
|                        r2). 
|-callsMockSource(<Mockito: Object mock(Class)>/test/0). !\label{line:facts-cms}!
|MockSourceMethod(<Mockito: Object mock(Class)>). !\label{line:facts-msm}!
|CallGraphEdge(_, <Mockito: Object mock(Class)>/test/0, _, !\label{line:facts-cge}!
|              <Mockito: Object mock(Class)>). 
\end{lstlisting}

\begin{lstlisting}[basicstyle=\ttfamily, caption={This example illustrates a field array container holding mock objects from \textit{setup()} in \texttt{NodeListIteratorTest.java}.},
basicstyle=\ttfamily,language = Java, framesep=4.5mm, framexleftmargin=1.0mm, captionpos=b, label=lis:container, escapechar=|, morekeywords={@Test}]
// Node array to be filled with mock Node instances
private Node[] nodes;
@Test
protected void setUp() throws Exception {
	// create mock Node Instances and 
	// fill Node[] to be used by test cases
	final Node node1 = createMock(Element.class);
	final Node node2 = createMock(Element.class);
	final Node node3 = createMock(Text.class);
	final Node node4 = createMock(Element.class);
	nodes = new Node[] {node1, node2, node3, node4}; |\label{line:storeMocksInArray}|
	// ...
}
\end{lstlisting}

\section{Extensions: arrays and collections} 

While we were designing \textsc{MockDetector}, we observed several cases where developers store mock objects in arrays and collections. Listing~\ref{lis:container} presents method \textit{setUp()} in class \texttt{NodeListIteratorTest} from commons-collections-4.4, where line \ref{line:storeMocksInArray} puts the mock \texttt{Node} objects in the array-typed field \texttt{nodes}. This field is later used in test cases. When the flow function of the dataflow analysis encounters an assignment statement containing an array read or write, it first looks for values (local variables or field reference sources) on the opposite side of the assignment statement---the statement's destination or source, respectively---in the Jimple intermediate representation. It then checks whether any of these local variables or field references have been marked as mock objects in the analysis. If so, the tool marks the local variable or field reference representing the array as an array mock---it propagates the mockness to the array container.

Figure~\ref{fig:arrayMockIllustration} illustrates the process of identifying a mock-containing array, and Listing~\ref{lis:arrayIllustrationIR} displays the Jimple IR of the code in Figure~\ref{fig:arrayMockIllustration}. Our analysis reaches the mock API call on line~\ref{line:lis4line4}--\ref{line:lis4line6}, where it records that \texttt{\$r2} is a mock object---it creates a MockStatus abstraction object with mock bit set to 1 and associates that object with \texttt{\$r2}. The tool then handles the cast expression assigning to \texttt{r1} on line~\ref{line:lis4line7}, giving it the same MockStatus as \texttt{\$r2}. When the analysis reaches line~\ref{line:lis4line9}, it finds an array reference on the left hand side, along with \texttt{r1} stored in the array on the right-hand side of the assignment statement. At that point, it has a MockStatus associated with \texttt{r1}, with the mock bit turned on. It can now deduce that \texttt{\$r3} on the left-hand side is an array container which may hold a mock object. Therefore, \textsc{MockDetector}'s imperative static analysis associates \texttt{\$r3} with a MockStatus with mock-containing array bit (``arrayMock'') set to 1.


\begin{figure}
	\begin{lstlisting}[
	basicstyle=\ttfamily,language = Java, framesep=4.5mm, framexleftmargin=1.0mm, captionpos=b, escapechar=|, morekeywords={@Test}]
	//        mock: |\cmark|     mockAPI: |\cmark|
	Object object1 = createMock(Object.class);
	
	// arrayMock: |\cmark| |$\Leftarrow$| array-write    |~~|  mock: |\cmark|
	objects  |~~|           = new Object[]  |~|  { object1 };
	\end{lstlisting}
	
	\caption{Our static analysis also finds array mocks.}
	\label{fig:arrayMockIllustration}
	
\end{figure}

\begin{lstlisting}[basicstyle=\ttfamily, caption={Jimple Intermediate Representation for the array in Figure~\ref{fig:arrayMockIllustration}.},
basicstyle=\ttfamily, framesep=4.5mm, framexleftmargin=1.0mm, captionpos=b, label=lis:arrayIllustrationIR, escapechar=|, morekeywords={@Test, specialinvoke, virtualinvoke, staticinvoke, newarray}]
java.lang.Object r1, $r2;
java.lang.Object[] $r3;

$r2 = staticinvoke <org.easymock.EasyMock: |\label{line:lis4line4}|
						java.lang.Object createMock(java.lang.Class)>
						(class "java.lang.Object;"); |\label{line:lis4line6}|
r1 = (java.lang.Object) $r2; |\label{line:lis4line7}|
$r3 = newarray (java.lang.Object)[1]; |\label{line:lis4line8}|
$r3[0] = r1;  |\label{line:lis4line9}|
\end{lstlisting}


The declarative analysis uses analogous reasoning, but uses a relation \\ \texttt{isArrayLocalThatContainsMocks} instead of a bit in the MockStatus abstraction.



%----------------------------------------------------------------------
% END MATERIAL
% Bibliography, Appendices, Index, etc.
%----------------------------------------------------------------------

% Bibliography

% The following statement selects the style to use for references.  
% It controls the sort order of the entries in the bibliography and also the formatting for the in-text labels.
\bibliographystyle{plain}
% This specifies the location of the file containing the bibliographic information.  
% It assumes you're using BibTeX to manage your references (if not, why not?).
\cleardoublepage % This is needed if the "book" document class is used, to place the anchor in the correct page, because the bibliography will start on its own page.
% Use \clearpage instead if the document class uses the "oneside" argument
\phantomsection  % With hyperref package, enables hyperlinking from the table of contents to bibliography             
% The following statement causes the title "References" to be used for the bibliography section:
\renewcommand*{\bibname}{References}

% Add the References to the Table of Contents
\addcontentsline{toc}{chapter}{\textbf{References}}

\bibliography{uw-ethesis}
% Tip: You can create multiple .bib files to organize your references. 
% Just list them all in the \bibliogaphy command, separated by commas (no spaces).

% The following statement causes the specified references to be added to the bibliography even if they were not cited in the text. 
% The asterisk is a wildcard that causes all entries in the bibliographic database to be included (optional).
\nocite{*}
%----------------------------------------------------------------------

% Appendices

% The \appendix statement indicates the beginning of the appendices.
\appendix
% Add an un-numbered title page before the appendices and a line in the Table of Contents
\chapter*{APPENDICES}
\addcontentsline{toc}{chapter}{APPENDICES}
% Appendices are just more chapters, with different labeling (letters instead of numbers).
%\chapter[PDF Plots From Matlab]{Matlab Code for Making a PDF Plot}
\label{AppendixA}
% Tip 4: Example (above) of how to get a shorter chapter title for the Table of Contents 
%======================================================================
\section{Using the Graphical User Interface}
Properties of Matab plots can be adjusted from the plot window via a graphical interface. Under the Desktop menu in the Figure window, select the Property Editor. You may also want to check the Plot Browser and Figure Palette for more tools. To adjust properties of the axes, look under the Edit menu and select Axes Properties.

To set the figure size and to save as PDF or other file formats, click the Export Setup button in the figure Property Editor.

\section{From the Command Line} 
All figure properties can also be manipulated from the command line. Here's an example: 
\begin{verbatim}
x=[0:0.1:pi];
hold on % Plot multiple traces on one figure
plot(x,sin(x))
plot(x,cos(x),'--r')
plot(x,tan(x),'.-g')
title('Some Trig Functions Over 0 to \pi') % Note LaTeX markup!
legend('{\it sin}(x)','{\it cos}(x)','{\it tan}(x)')
hold off
set(gca,'Ylim',[-3 3]) % Adjust Y limits of "current axes"
set(gcf,'Units','inches') % Set figure size units of "current figure"
set(gcf,'Position',[0,0,6,4]) % Set figure width (6 in.) and height (4 in.)
cd n:\thesis\plots % Select where to save
print -dpdf plot.pdf % Save as PDF
\end{verbatim}

% GLOSSARIES (Lists of definitions, abbreviations, symbols, etc. provided by the glossaries-extra package)
% -----------------------------
\printglossaries
\cleardoublepage
\phantomsection		% allows hyperref to link to the correct page

%----------------------------------------------------------------------
\end{document} % end of logical document
